\documentclass{article}
\usepackage{amsmath}
\usepackage{amsthm}
\usepackage{amssymb}
\usepackage[retainorgcmds]{IEEEtrantools}
\usepackage{tikz-cd}

\begin{document}
\title{Binary Quintic Forms, Quintic Rings \& Sextic Resolvents}
\author{Dan Fess}
\date{5th February 2021}
\maketitle

\newtheorem{theorem}{Theorem}
\newtheorem{lemma}{Lemma}[section]
\newtheorem{prop}[lemma]{Proposition}
\newtheorem{corollary}[lemma]{Corollary}
\newtheorem{conjecture}{Conjecture}
\newtheorem{definition}[lemma]{Definition}

\section{Introduction}

By a construction of Birch and Merriman, we can attach to a binary quintic form $f$ a quintic ring $R_f$.  Not all quintic rings are produced this way; the forms describe a certain subclass of quintic rings.  What is special about this subclass?  Which quintic rings are these?  We can also associate to a quintic ring a sextic resolvent ring; perhaps there is something special about the sextic resolvent?

Another problem we can hope to understand is the number of $GL_2(\mathbb{Z})$-classes of binary quintic forms of bounded discriminant.  The collection of all quintic rings is described by the space $\mathbb{Z}^4 \otimes \wedge^2 \mathbb{Z}^5$, which naturally has an action of $\Gamma = GL_4(\mathbb{Z}) \times GL_5(\mathbb{Z})$.  If we can find a discriminant-preserving, orbit-preserving map from binary quintic forms to this space, then we can hope to translate our knowledge of orbits in $\mathbb{Z}^4 \otimes \wedge^2 \mathbb{Z}^5$ to study classes of binary quintics.

\subsection{Results}

Denote a generic element $f \in (Sym^5 \mathbb{Z}^2)^*$ by $f = f_0 x^5 + f_1 x^4 y + f_2 x^3 y^2 + f_3 x^2 y^3 + f_4 x y^4 + f_5 y^5$.

\begin{theorem}
There is a map $\Phi : (Sym^5 \mathbb{Z}^2)^* \to \mathbb{Z}^4 \otimes \wedge^2 \mathbb{Z}^5$ which respects the two constructions of quintic rings.  Furthermore, it is discriminant- and orbit-preserving.  Explictly:\\

$\Phi(f) = \begin{pmatrix}
0 & t_3 & - t_2 & t_1 & 0\\
- t_3 & 0 & -  f_0 t_1 -  f_1 t_2 & -  f_2 t_2 -  f_3 t_3 & - t_4\\
t_2 &  f_0 t_1 +  f_1 t_2 & 0 & -  f_4 t_3 -  f_5 t_4 & t_3\\
- t_1 &  f_2 t_2 +  f_3 t_3 &  f_4 t_3 +  f_5 t_4 & 0 & - t_2\\
0 & t_4 & - t_3 & t_2 & 0\\
\end{pmatrix}$\\
\end{theorem}

For $A \in \mathbb{Z}^4 \otimes \wedge^2 \mathbb{Z}^5$, denote the associated based quintic ring by $R(A)$ and the based sextic resolvent ring by $S(A)$.  This theorem tells us that $R_f = R(\Phi(f))$, but the more interesting ramification is that it gives us access to the sextic resolvent ring $S(\Phi(f))$, which we denote by $S_f$.  We can explicitly compute its multiplicative structure, and find out that the sextic resolvent rings coming from binary quintic forms have the following unusual structure:

\begin{definition} \label{dual duogenic}
Let $(R,S)$ be a quintic ring / sextic resolvent pair, with both rings of non-zero discriminant.  Consider $S$ as a based ring, with basis denoted by $\{1,\beta_1,\ldots,\beta_5\}$.  Denote the dual basis with respect to the trace pairing by $\{\beta_0^*,\beta_1^*,\ldots,\beta_5^*\}$, i.e. $\beta_i^* \in S \otimes \mathbb{Q}$ such that $Tr(\beta_i^* \beta_j) = \delta_{ij}$.

Then, we say that the based ring $S$ is \emph{dual-duogenic} if it has the following property:
\begin{equation}
(\beta_2,\beta_3,\beta_4) \equiv 8 \, Disc(R) \cdot ( (\beta_1^*)^2, 2 \beta_1^* \beta_5^*, (\beta_5^*)^2 ) \, \text{mod} \, \mathbb{Q}
\end{equation}

This can also be rephrased in terms of $Disc(S)$ using $Disc(S) = (16 \cdot Disc(R))^3$.

We also define a (non-based) ring $S$ to be dual-duogenic if it has a basis in which it is dual-duogenic.
\end{definition}

\begin{theorem} \label{sextic structure}
Let $f$ be a binary quintic form of non-zero discriminant $\Delta(f)$.  Then, the based ring $S_f$ is dual-duogenic.
\end{theorem}

Nakagawa proved that if $f$ and $g$ are $GL_2(\mathbb{Z})$-equivalent binary quintic forms, their associated quintic rings $R_f$ and $R_g$ are isomorphic, but have different canonical bases; the $GL_2(\mathbb{Z})$ action induces this change of basis of the quintic ring, and can equally be viewed as a $GL_2(\mathbb{Z})$ action on $R_f / \mathbb{Z} \simeq \mathbb{Z}^4$.  Similarly, in the sextic resolvent ring, the $GL_2(\mathbb{Z})$ action induces a change of basis and a corresponding change of dual basis, which respect the special property of dual-duogenicity.

\begin{theorem}
Let $f,g$ be binary quintic forms with $\gamma \in GL_2(\mathbb{Z})$ such that $g = \gamma \cdot f$.  Then, $S_f$ and $S_g$ have different canonical bases but are isomorphic in a way that respects dual-duogenicity, i.e. $\gamma$ acts linearly on $sp \{\beta_1^*, \beta_5^*\}$ and quadratically on $sp \{\beta_2,\beta_3,\beta_4\}$.
\end{theorem}

We would like to exactly pin down the quintic ring / sextic resolvent pairs $(R,S)$ given by binary quintics.  Thus, we need to know if this property is sufficient for the pair to have come from a binary quintic.  With an extra condition on the alternating matrix $A$, there is indeed a basis of $R$ such that it comes from a binary quintic form:
\begin{theorem}
Let $A \in \mathbb{Z}^4 \otimes \wedge^2 \mathbb{Z}^5$, with associated based quintic ring and based sextic resolvent ring $(R(A),S(A))$.

Let $H \leqslant SL_5(\mathbb{Z})$ be the group of all matrices of the following form:

\begin{equation}
\begin{pmatrix}
1 & 0 & 0 & 0 & 0 \\
* & 1 & 0 & 0 & * \\
* & 0 & 1 & 0 & * \\
* & 0 & 0 & 1 & * \\
0 & 0 & 0 & 0 & 1
\end{pmatrix}
\end{equation}

Then, $A$ is an $SL_4(\mathbb{Z}) \times H$ translate of $\Phi(f)$ for some integral binary quintic form $f$ if and only if the following two conditions hold:
\begin{itemize}
\item $A(e_1,e_5) = 0$
\item $S(A)$ is dual-duogenic
\end{itemize}

In this case, there is a basis of $R(A)$ arising from the binary quintic $f$.
\end{theorem}

Furthermore, by studying the action of various groups involved in this picture, we can upgrade this theorem so that it concerns classes of binary quintic forms.  This theorem involves the notion of a choice of dual-duogenization, which will be defined later, in Section \ref{sextic section}; in short, it recognises the different ways in which a sextic ring can have a dual-duogenic basis.
\begin{theorem}
There is a bijection between the following two sets:
\begin{equation}
GL_2(\mathbb{Z}) \backslash 
(Sym^5 \mathbb{Z}^2)^* \leftrightarrow
\begin{cases}
(R,S), \text{R basis-free} \\
\text{S dual-duogenic with a fixed dual-duogenization} \\
\text{with } g(\beta_1^*,\beta_5^*) = 0 
\end{cases}
 / \sim
\end{equation}
given by $f \mapsto$ the class of $(R_f, S_f)$, where the $\sim$ denotes isomorphism of $R$ and dual-duogenic isomorphism of $S$, and where $g$ is the fundamental alternating map $\wedge^2 \tilde{S} \to \tilde{R}$.
\end{theorem}

The other direction in which we would like to take this work is to count classes of binary quintic forms of bounded discriminant.  The number of $\Gamma$-orbits in $\mathbb{Z}^4 \otimes \wedge^2 \mathbb{Z}^5$ of bounded discriminant is understood, and the map $\Phi$ is discriminant- and orbit-preserving, so if we understand $\Phi$ well then we can count classes of binary quintics.  Specifically, we need to know how many classes of binary quintics can possibly land in one $\Gamma$-orbit in $\mathbb{Z}^4 \otimes \wedge^2 \mathbb{Z}^5$.  The following result goes some way towards this:

\begin{theorem} [In Progress]
Given a binary quintic $f \in (Sym^5 \mathbb{Z}^2)^*$, there is an associated Segre cubic threefold $V_f \subseteq \mathbb{P}^4$.  The Fano variety of lines on $V_f$ has a distinguished 2-dimensional component $D \subseteq \mathbb{P}^9$.  Certain integral points on the cone over $D$ in $\mathbb{A}^{10}$ correspond to classes of forms which land in the same $\Gamma$-orbit as $\Phi(f)$.  Furthermore, we can provide explicit polynomials cutting out a 2-dimensional subvariety of $\mathbb{A}^{10}$, whose integral points are precisely those in question.
\end{theorem}

This theorem is proved, except for the statement that the subvariety in question is of dimension 2.   I have only proved this in the case of $f = x^5 + y^5$.  In the general case, I currently know that it is of dimension at most 2.

\section{Preliminaries regarding $n$-ic rings}

To add:
\begin{itemize}
\item Cubic and quartic rings ( + resolvents)
\item More detail on quintic rings / sextic resolvents
\item $n$-ic rings from binary $n$-ic forms
\item Relevant material on $n$-ic rings in general
\end{itemize}

\section{Maps from $(Sym^5 \mathbb{Z}^2)^* \to \mathbb{Z}^4 \otimes \wedge^2 \mathbb{Z}^5$}

The maps below mimic closely the maps from binary quartics to $\mathbb{Z}^2 \otimes Sym^2 \mathbb{Z}^3$ defined by Wood.

Denote a generic element $f \in (Sym^5 \mathbb{Z}^2)^*$ by $f = f_0 x^5 + f_1 x^4 y + f_2 x^3 y^2 + f_3 x^2 y^3 + f_4 x y^4 + f_5 y^5$.

\begin{theorem} \label{Phi}
The following map $\Phi: (Sym^5 \mathbb{Z}^2)^* \to \mathbb{Z}^4 \otimes \wedge^2 \mathbb{Z}^5$ respects the two constructions of based quintic rings:\\

$\Phi(f) = \begin{pmatrix}
0 & t_3 & - t_2 & t_1 & 0\\
- t_3 & 0 & -  f_0 t_1 -  f_1 t_2 & -  f_2 t_2 -  f_3 t_3 & - t_4\\
t_2 &  f_0 t_1 +  f_1 t_2 & 0 & -  f_4 t_3 -  f_5 t_4 & t_3\\
- t_1 &  f_2 t_2 +  f_3 t_3 &  f_4 t_3 +  f_5 t_4 & 0 & - t_2\\
0 & t_4 & - t_3 & t_2 & 0\\
\end{pmatrix}$\\

\end{theorem}
\begin{proof}
Computation of the $SL_5$-invariants leads to the multiplicative structure of $R(\Phi(f))$, which is seen to match that of the ring $R_f$ defined by Birch and Merriman.
\end{proof}

However, the derivation of this map was not entirely guesswork.  It relies on noticing that the sub-pfaffians can be taken to be quite simple quadrics, as explained by the following results, which massively simplifies brute forcing a candidate map $\Phi$.
\begin{prop} \label{n points}
Let $f$ be a binary $n$-ic form of non-zero discriminant.  Recall that we may attach a set of $n$ points in $\mathbb{P}^{n-2}$ to the based $n$-ic ring $R_f$, denoted by $X_{R_f}$.   Then, $X_{R_f} = \{ ( x^{n-2} : x^{n-3} y : \ldots : x y^{n-3} : y^{n-2} ) : f(x, y) = 0 \}$.
\end{prop}

\begin{proof}
(Insert this in an earlier section with preliminary results on $n$-ic rings.)
\end{proof}

\begin{prop} \label{sub-pfaffians}
Let $f$ have non-zero discriminant and let $R_f$ be given by $A_f \in \mathbb{Z}^4 \otimes \wedge^2 \mathbb{Z}^5$.  The sub-pfaffians of $A_f$ span the space of quadrics vanishing on $X_{R_f}$.  One such basis for this space is seen to be:
\begin{IEEEeqnarray}{rCl}
Q_1 & = & t_1 t_3 - t_2^2\\
Q_2 & = & t_1 t_4 - t_2 t_3\\
Q_3 & = & t_2 t_4 - t_3^2\\
Q_4 & = &  f_0 t_1 t_2 +  f_1 t_2^2 +  f_2 t_2 t_3 +  f_3 t_3^2 +  f_4 t_3 t_4 +  f_5 t_4^2\\
Q_5 & = &  f_0 t_1^2 +  f_1 t_1 t_2 +  f_2 t_2^2 +  f_3 t_2 t_3 +  f_4 t_3^2 +  f_5 t_3 t_4
\end{IEEEeqnarray}

\end{prop}

\begin{proof}
It is known from Higher Composition Laws IV that the sub-pfaffians of non-degenerate $A \in \mathbb{Z}^4 \otimes \wedge^2 \mathbb{Z}^5$ span the space of quadrics vanishing on $X_{R(A)}$.

For the latter part of the claim, we use Proposition \ref{n points}.  The forms $Q_1,Q_2,Q_3$ cut out the rational normal curve in $\mathbb{P}^3$, which contains $X_{R_f}$.  The forms $Q_4, Q_5$, when evaluated on the rational normal curve, are then rephrasings of the equations $x \cdot f(x,y) = 0$ and $y \cdot f(x,y) = 0$.  These are easily seen to span the space of quadrics vanishing on $X_{R_f}$ - by explicit computation, or by a dimension count.
\end{proof}

\begin{corollary} \label{disc}
The map $\Phi$ is discriminant-preserving.
\end{corollary}
\begin{proof}
For $A \in \mathbb{Z}^4 \otimes \wedge^2 \mathbb{Z}^5$, the discriminant of $R(A)$ was shown by Bhargava to be $Disc(A)$.  Similarly, Birch and Merriman showed that the discriminant of $R_f$ is $\Delta(f)$.  The result then follows from Theorem \ref{Phi}.
\end{proof}

\begin{theorem} \label{GL2}
If $\gamma \in GL_2(\mathbb{Z})$, then $\Phi(\gamma \cdot f)$ and $\Phi(f)$ are $GL_4(\mathbb{Z}) \times GL_5(\mathbb{Z})$-equivalent.

Explicitly, define $\sigma :  GL_2(\mathbb{Z}) \to GL_4(\mathbb{Z}) \times GL_5(\mathbb{Z})$ as follows:
\begin{IEEEeqnarray}{rCl}
\gamma & = & \begin{pmatrix} a & b \\ c & d \end{pmatrix}\\
\sigma(\gamma) & = & (\psi(\gamma),\rho(\gamma))\\
\psi(\gamma) & = & (ad-bc) \begin{pmatrix}
a^3 & a^2 b & a b^2 & b^3 \\
3 a^2 c & 2 a b c + a^2 d & 2 a b d + b^2 c & 3 b^2 d\\
3 a c^2 & 2 a c d + b c^2 & 2 b c d + a d^2 & 3 b d^2\\
c^3 & c^2 d & c d^2 & d^3
\end{pmatrix}\\
\rho(\gamma) & = & \begin{pmatrix}
d & 0 & 0 & 0 & - c \\
0 & a^2 & 2ab & b^2 & 0 \\
0 & ac & bc+ad & bd & 0 \\
0 & c^2 & 2cd & d^2 & 0 \\
- b & 0 & 0 & 0 & a
\end{pmatrix}
\end{IEEEeqnarray}

Recall that $H \leqslant SL_5(\mathbb{Z})$ is the group of all matrices of the form:

\begin{equation}
\begin{pmatrix}
1 & 0 & 0 & 0 & 0 \\
* & 1 & 0 & 0 & * \\
* & 0 & 1 & 0 & * \\
* & 0 & 0 & 1 & * \\
0 & 0 & 0 & 0 & 1
\end{pmatrix}
\end{equation}

Then, there exists $h \in H$ such that $\Phi( \gamma \cdot f) = (1,h) \cdot \sigma ( \gamma ) \cdot \Phi(f)$.

\end{theorem}

\begin{proof}
This can be checked by explicit computation, but the derivation of the result in the first place is illuminating.

Let $g = \gamma \cdot f$.  Nakagawa proved that $R_f = R_g$, with their bases related (mod $\mathbb{Z}$) by $\psi(\gamma)$.  Hence, if $\Phi(g) = ( \alpha , \beta ) \cdot \Phi(f)$ for some $( \alpha , \beta ) \in \Gamma$, then $\alpha = \psi ( \gamma )$.

To find a candidate for $\beta$, we consider the action on sub-pfaffians.  Denote the five $4 \times 4$ signed sub-pfaffians of $\Phi(f)$ by $P_{1,f}, \ldots, P_{5,f}$, and do the same for $g$.  If $\Phi(g) = ( \alpha, \beta ) \cdot \Phi(f)$ then Higher Composition Laws IV tells us that:

\begin{equation}
\begin{pmatrix}
P_{1,g} \\ P_{2,g} \\ P_{3,g} \\ P_{4,g} \\ P_{5,g}
\end{pmatrix}
=
(\det \beta ) (\beta^{-1})^t
\begin{pmatrix}
\alpha P_{1,f} \alpha^t \\\alpha  P_{2,f} \alpha^t \\ \alpha P_{3,f} \alpha^t \\ \alpha P_{4,f} \alpha^t \\ \alpha P_{5,f} \alpha^t
\end{pmatrix}
\end{equation}

Using $\alpha = \psi ( \gamma )$, we compute this all and see that $\beta$ would have to be of the form $h \cdot \rho( \gamma )$ for a unique $h$.

Then, we check with this $\beta$ that indeed $\Phi(g) = (\alpha,\beta) \cdot \Phi(f)$.
\end{proof}

There is a relative $\Phi'$ of $\Phi$, which is $GL_2(\mathbb{Z})$-equivariant on the nose (i.e. not up to some $h \in H$), but is only defined for integer-matrix quintic forms.  They are related in the sense that $\Phi(f)$ and $\Phi'(f)$ are $GL_5(\mathbb{Z})$-translates:
\begin{prop}
There exists $\Phi' : Sym^5 \mathbb{Z}^2 \to \mathbb{Z}^4 \otimes \wedge^2 \mathbb{Z}^5$ and $\sigma: GL_2(\mathbb{Z}) \to GL_4(\mathbb{Z}) \times GL_5(\mathbb{Z})$ as above, such that $\Phi'(\gamma \cdot f) = \sigma(\gamma) \cdot \Phi'(f)$.  We also have that $R(\Phi'(f)) = R_f$ as based rings.

In full detail, $\Phi(f) = t_1 A_1 + t_2 A_2 + t_3 A_3 + t_4 A_4$ for:

\begin{IEEEeqnarray}{rCl}
A_1 & = & \begin{pmatrix}
0 & 0 & 0 & 1 & 0 \\
& 0 & -  f_0 & - \frac{2 f_1}{5} & 0 \\
& & 0 & - \frac{ f_2}{10} & 0 \\
& & & 0 & 0 \\
& & & & 0
\end{pmatrix} \\
A_2 & = & \begin{pmatrix}
0 & 0 & - 1 & 0 & 0 \\
& 0 & - \frac{3  f_1}{5} & - \frac{3  f_2}{5} & 0 \\
& & 0 & - \frac{3  f_3}{10} & 0 \\
& & & 0 & -1 \\
& & & & 0
\end{pmatrix} \\
A_3 & = & \begin{pmatrix}
0 & 1 & 0 & 0 & 0 \\
& 0 & - \frac{3  f_2}{10} & - \frac{3  f_3}{5} & 0 \\
& & 0 & - \frac{3  f_4}{5} & 1 \\
& & & 0 & 0 \\
& & & & 0
\end{pmatrix} \\
A_4 & = & \begin{pmatrix}
0 & 0 & 0 & 0 & 0 \\
& 0 & - \frac{ f_3}{10} & - \frac{2  f_4}{5} & -1 \\
& & 0 & -  f_5 & 0 \\
& & & 0 & 0 \\
& & & & 0
\end{pmatrix}
\end{IEEEeqnarray}

\end{prop}

\begin{proof}
We aim to find a $GL_5(\mathbb{Z})$-translate of $\Phi(f)$, which respects the $GL_2(\mathbb{Z})$ action on the nose and does not need slight adjustment by $h \in H$.  Thus, we look for a polynomial map $P(f) \in H$ such that $\Phi'(f) = P(f) \cdot \Phi(f)$ is $GL_2(\mathbb{Z})$-equivariant, i.e. we want $\Phi'( \gamma \cdot f ) = \sigma( \gamma ) \cdot \Phi'( f )$.

Our work from Theorem \ref{GL2} in fact tells us that $\Phi( \gamma \cdot f) = (1,h(\gamma,f)) \cdot \sigma(\gamma) \cdot \Phi(f)$, for $h(\gamma,f)$ polynomial in $\gamma$ and $f$.  We calculate:
\begin{IEEEeqnarray}{rCl}
\Phi'( \gamma \cdot f ) & = & P( \gamma \cdot f ) \cdot \Phi( \gamma \cdot f ) \\
& = & P( \gamma \cdot f ) \cdot (1,h(\gamma,f)) \cdot \sigma(\gamma) \cdot \Phi(f) \\
& = & P( \gamma \cdot f ) \cdot (1,h(\gamma,f)) \cdot \sigma(\gamma) \cdot P(f)^{-1} \Phi'(f) \\
& = & ( \psi(\gamma), P( \gamma \cdot f ) h(\gamma,f) \rho(\gamma) P(f)^{-1} ) \cdot \Phi'(f)
\end{IEEEeqnarray}

Thus, we see that we need $P$ such that $P( \gamma \cdot f ) h(\gamma,f) \rho(\gamma) P(f)^{-1} = \rho(\gamma)$.  From explicit calculation, the following $P$ arises and indeed does the job, leading to $\Phi'(f)$:
\begin{equation}
P(f) =
\begin{pmatrix}
1 & 0 & 0 & 0 & 0\\ - \frac{2}{5}  f_1 & 1 & 0 & 0 & \frac{2}{5}  f_2 \\ - \frac{1}{10} f_2 & 0 & 1 & 0 & \frac{1}{10} f_3 \\ - \frac{2}{5} f_3 & 0 & 0 & 1 & \frac{2}{5} f_4 \\ 0 & 0 & 0 & 0 & 1
\end{pmatrix}
\end{equation}
\end{proof}

This map is only defined for integer-matrix forms, but there is a third, related map which respects the action of $GL_2(\mathbb{Z})$ and is defined for all binary quintic forms.  Inspired by the analogous map in the case of binary quartics, we can map to a quotient of $\mathbb{Z}^4 \otimes \wedge^2 \mathbb{Z}^5$:
\begin{corollary}
Define $\bar{\Phi} : Sym^5 \mathbb{Z}^2 \to H \backslash (\mathbb{Z}^4 \otimes \wedge^2 \mathbb{Z}^5)$ by $\bar{\Phi}(f) = H \Phi(f)$.  Then $\bar{\Phi}(\gamma \cdot f) = \sigma(\gamma) \cdot \bar{\Phi}(f)$.
\end{corollary}
\begin{proof}
Follows from Theorem \ref{GL2}.  [Note that this claim is well-defined: $\sigma(\gamma) \cdot \bar{\Phi}(f)$ is a left $H$-coset because $\sigma(\gamma) H \sigma(\gamma)^{-1} = H$.]
\end{proof}

\begin{corollary}
The maps $\Phi'$ and $\bar{\Phi}$ are both discriminant-preserving.
\end{corollary}
\begin{proof}
The based quintic ring in question is unchanged, so the same proof as that of Corollary \ref{disc} applies.
\end{proof}








\section{The sextic resolvent ring $S_f$} \label{sextic section}

Binary quartic forms have the special property that they describe quartic rings whose cubic resolvent is monogenic.  Given a binary quintic form, with associated quintic ring and sextic resolvent, what special properties does the resolvent ring have?

For binary quartic forms, the basis of the cubic resolvent ring depends only on the $GL_2(\mathbb{Z})$-orbit of the binary quartic in question.  This is because the relevant map $GL_2(\mathbb{Z}) \to GL_2(\mathbb{Z}) \times GL_3(\mathbb{Z})$ has trivial $GL_2(\mathbb{Z})$ component.  This means that the structure of the cubic resolvent can be understood in terms of the $GL_2$ invariants of the binary quartic form.

For binary quintic forms, since the map $\sigma$ lands non-trivially in $GL_5(\mathbb{Z})$, this is not the case; just as the structure coefficients of the based ring $R_f$ depend on $f$ and not just its equivalence class, so the same will be true for the sextic resolvent $S_f = S(\Phi(f))$.

Yet, the block matrix form of $\rho(\gamma)$ is intriguing, and suggests there may be special structure in the resolvent ring; indeed, this has to do with the relation between $\{ \beta_1^*, \beta_5^* \}$ and $\{ \beta_2, \beta_3, \beta_4 \}$.

\subsection{The structure of the sextic resolvent ring}

Recall the definition of dual-duogenicity:
\begin{definition} \label{dual duogenic2}
Let $(R,S)$ be a quintic ring / sextic resolvent pair, with both rings of non-zero discriminant.  Consider $S$ as a based ring, with basis denoted by $\{1,\beta_1,\ldots,\beta_5\}$.  Denote the dual basis with respect to the trace pairing by $\{\beta_0^*,\beta_1^*,\ldots,\beta_5^*\}$, i.e. $\beta_i^* \in S \otimes \mathbb{Q}$ such that $Tr(\beta_i^* \beta_j) = \delta_{ij}$.

Then, we say that the based ring $S$ is \emph{dual-duogenic} if it has the following property:
\begin{equation}
(\beta_2,\beta_3,\beta_4) \equiv 8 \, Disc(R) \cdot ( (\beta_1^*)^2, 2 \beta_1^* \beta_5^*, (\beta_5^*)^2 ) \, \text{mod} \, \mathbb{Q}
\end{equation}

This can also be rephrased in terms of $Disc(S)$ using $Disc(S) = (16 \cdot Disc(R))^3$.

We also define a (non-based) ring $S$ to be dual-duogenic if it has a basis in which it is dual-duogenic.
\end{definition}

A few notes on this definition:
\begin{itemize}
\item That $S$ is based means that the $\beta_i$ are only determined mod $\mathbb{Z}$.  However, the $\beta_i^*$ will then be completely determined.  Thus, the notion of dual-duogenicity is well-defined.
\item Since the basis of $S$ is determined only mod $\mathbb{Z}$, it wouldn't make sense for this theorem to be stated on the nose with an equality.
\item But, then why is it stated mod $\mathbb{Q}$ and not mod $\mathbb{Z}$?  We will see that, for $S = S_f$, the three expressions $8 \Delta(f) \cdot ((\beta_1^*)^2, 2 \beta_1^* \beta_5^*, (\beta_5^*)^2)$ may not lie in $S$, but some $\mathbb{Q}$-translate of each of them does.  These will in turn be congruent to $\beta_2, \beta_3, \beta_4$ mod $\mathbb{Z}$.
\end{itemize}


\begin{theorem} \label{sextic structure2}
Let $f$ be a binary quintic form of non-zero discriminant $\Delta(f)$.  Then, the based ring $S_f$ is dual-duogenic.
\end{theorem}
\begin{proof}
Proof is by computation of the $SL_4$-invariants of $\Phi(f)$, from which the multiplicative structure of $S_f$ can be computed as explained in Higher Composition Laws IV.  [See Appendix for multiplication table.]
\end{proof}

[Note:   If we had a different way to prove the Segre cubic = trace cubed property of Theorem \ref{trace cubed}, that would result in an alternative proof of the above result.  Currently, the proof of Theorem \ref{trace cubed} also relies on the multiplication table of $S_f$, though it is true for all $S(A)$.]

Using this result, we can say more about $x,y$:
\begin{lemma}
$Tr(x^3) = Tr(x^2 y) = Tr(x y^2) = Tr(y^3) = 0$.
\end{lemma}
\begin{proof}
Since $Tr(\beta_i \beta_j^*) = 0$ for $i \neq j$ and $Tr(\beta_j^*) = 0$ for $j \neq 0$, choosing $i \in \{2,3,4\}$, $j \in \{1,5\}$ and applying Theorem \ref{sextic structure2} does the job.
\end{proof}

\begin{lemma} \label{5th powers}
$(8 \Delta(f))^2 \cdot Tr(x^i y^j) \in \mathbb{Z}$ for $i,j \geq 0, i + j = 5$.
\end{lemma}
\begin{proof}
For $i,j \in \{2,3,4\}$, $k \in \{1,5\}$, $\mathbb{Z} \ni d_{ij}^k = Tr(\beta_i \beta_j \beta_k^*)$ and this expression is invariant under translating $\beta_i, \beta_j$ by elements of $\mathbb{Q}$.  Hence:
\begin{IEEEeqnarray}{rClCl}
d_{22}^1 & = & (8 \Delta(f))^2 & \cdot & Tr(x^5)\\
d_{22}^5 & = & (8 \Delta(f))^2 & \cdot & Tr(x^4 y)\\
d_{23}^1 & = & 2 (8 \Delta(f))^2 & \cdot & Tr(x^4 y)\\
d_{23}^5 & = & 2 (8 \Delta(f))^2 & \cdot & Tr(x^3 y^2)\\
d_{24}^1 & = & (8 \Delta(f))^2 & \cdot & Tr(x^3 y^2)\\
d_{24}^5 & = & (8 \Delta(f))^2 & \cdot & Tr(x^2 y^3)\\
d_{33}^1 & = & 4 (8 \Delta(f))^2 & \cdot & Tr(x^3 y^2)\\
d_{33}^5 & = & 4 (8 \Delta(f))^2 & \cdot & Tr(x^2 y^3)\\
d_{34}^1 & = & 2 (8 \Delta(f))^2 & \cdot & Tr(x^2 y^3)\\
d_{34}^5 & = & 2 (8 \Delta(f))^2 & \cdot & Tr(x y^4)\\
d_{44}^1 & = & (8 \Delta(f))^2 & \cdot & Tr(x y^4)\\
d_{44}^5 & = & (8 \Delta(f))^2 & \cdot & Tr(y^5)
\end{IEEEeqnarray}
\end{proof}

\begin{lemma} \label{recovery}
$Tr ( (u \beta_5^* - v \beta_1^*)^5) = - 10 f(u,v)$
\end{lemma}
\begin{proof}
Finding the structure coefficients $d_{ij}^k$ in the Appendix, we see that for $i,j \in \{2,3,4\}, k \in \{1,5\}$, they are all integer multiples of some $f_i$:
\begin{IEEEeqnarray}{rClCrCl}
d_{22}^1 & = & 10 f_5 & \, || \, & d_{22}^5 & = & - 2 f_4\\
d_{23}^1 & = & - 4 f_4 & \, || \, & d_{23}^5 & = & 2 f_3\\
d_{24}^1 & = & f_3 & \, || \, & d_{24}^5 & = & - f_2\\
d_{33}^1 & = & 4 f_3 & \, || \, & d_{33}^5 & = & - 4 f_2\\
d_{34}^1 & = & - 2 f_2 & \, || \, & d_{34}^5 & = & 4 f_1\\
d_{44}^1 & = & 2 f_1 & \, || \, & d_{44}^5 & = & - 10 f_0
\end{IEEEeqnarray}
Lemma \ref{5th powers} then completes the proof.
\end{proof}

\begin{corollary}
A based sextic ring $S$ can arise from at most one binary quintic form as $S_f = S(\Phi(f))$.
\end{corollary}
\begin{proof}
If $S = S_f$ for some $f$, then $f$ is encoded in $\beta_1^*, \beta_5^*$ as in Lemma \ref{recovery}.
\end{proof}

\subsection{An associated Segre cubic threefold}

There is a variety called a Segre cubic 3-fold attached to an element $A \in \mathbb{C}^4 \otimes \wedge^2 \mathbb{C}^5$.  This is work of Seok Hyeong (Sean) Lee [All original work?  If not, which parts are due to Seok Hyeong?].  The construction is as follows:

Given a quadruple $A = (A_1, A_2, A_3, A_4)$ of skew-symmetric $5 \times 5$ matrices, for each point $x = (x_1:x_2:x_3:x_4)$ in $\mathbb{P}^3$, the $5 \times 5$ skew-symmetric matrix $x_1 A_1 + x_2 A_2 + x_3 A_3 + x_4 A_4$ has even rank.  We denote this matrix by $A(x)$.  For generic $x \in \mathbb{P}^3$, it has rank 4, but for bad $x$ - which turn out to be the five points of $X_{R(A)}$ - its rank will be 2.  Consider the following subvariety of $\mathbb{P}^3 \times \mathbb{P}^4$: $V = \{ (x,y) : y \in ker(A(x)) \}$.  When we consider the map $V \to \mathbb{P}^3$, away from the five bad points identified, there is a unique point above each point $x \in \mathbb{P}^3$, since $dim(ker(A(x))) = 1$ generically.  At the five bad points, by virtue of $dim(ker(A(x))) = 3$, there is a plane above each such point.  So, $V$ is $\mathbb{P}^3$ blown up at five points.  If we now consider the image of $V$ in $\mathbb{P}^4$, it turns out that this is a Segre cubic 3-fold, which is a dimension 3 cubic variety with ten singularities, which are all nodal.

\begin{theorem} [S. H. Lee?] \label{segre equation}
Let $A \in \mathbb{C}^4 \otimes \wedge^2 \mathbb{C}^5$, $y,z \in \mathbb{C}^5$ with $y = (y_1, y_2, y_3, y_4, y_5)^t$ and $z = (z_1, z_2, z_3, z_4, z_5)^t$.  The determinant $A_1 y \wedge A_2 y \wedge A_3 y \wedge A_4 y \wedge z$ factors as $\langle y,z \rangle \cdot F_A(y)$, where $\langle y,z \rangle = \sum y_i z_i$ is the usual bilinear dot product and $F_A(y)$ is a cubic form in $y$.  Then, $F_A$ cuts out the Segre cubic threefold associated to $A$.
\end{theorem}

\begin{lemma} \label{segre action}
Let $(g,h) \in GL_4(\mathbb{C}) \times GL_5(\mathbb{C})$, and let $A' = (g,h) \cdot A$.  Then:
\begin{equation}
F_{A'}(y) = \det(g) \det(h) \, F_A(h^t y)
\end{equation}
\end{lemma}

\begin{proof}
\begin{IEEEeqnarray}{rCl}
\langle y,z \rangle \cdot F_{A'}(y) & = & A'_1 y \wedge A'_2 y \wedge A'_3 y \wedge A'_4 y \wedge z \\
& = & \det(g) \cdot \nonumber \\
& & \quad (hAh^t)_1 y \wedge (hAh^t)_2 y \wedge (hAh^t)_3 y \wedge (hAh^t)_4 y \wedge z \\
& = & \det(g) \det(h) \cdot \nonumber \\
& & \quad (Ah^t)_1 y \wedge (Ah^t)_2 y \wedge (Ah^t)_3 y \wedge (Ah^t)_4 y \wedge h^{-1} z \\
& = & \det(g) \det(h) \langle h^t y, h^{-1} z \rangle \cdot  F_A(h^t y) \\
& = & \det(g) \det(h) \langle y, z \rangle \cdot F_A(h^t y)
\end{IEEEeqnarray}
\end{proof}

Before we state how the form $F_A$ is related to the arithmetic of the sextic resolvent ring, we take a moment to note that the formulae of Higher Composition Laws IV define a commutative, associative sextic $\mathbb{C}$-algebra $S(A)$ when we extend them to $A \in \mathbb{C}^4 \otimes \wedge^2 \mathbb{C}^5$, because these properties are algebraic and $\mathbb{Z}^4 \otimes \wedge^2 \mathbb{Z}^5$ is Zariski dense in $\mathbb{C}^4 \otimes \wedge^2 \mathbb{C}^5$.  Furthermore, $SL_4(\mathbb{C}) \times SL_5(\mathbb{C})$ acts on the basis of $S(A) / \mathbb{C}$ in the same way that $SL_4(\mathbb{Z}) \times SL_5(\mathbb{Z})$ acts on the basis of $S(A) / \mathbb{Z}$ for integral $A$, because $SL_n(\mathbb{Z})$ is Zariski dense in $SL_n(\mathbb{C})$ [proof / source?].

Now we can state the meaning of $F_A$ in the sextic algebra $S(A)$:

\begin{theorem} \label{trace cubed}
Let $A, F_A, y$ be as in Theorem \ref{segre equation}, but with $Disc(A) \neq 0$.  The sextic resolvent algebra $S = S(A)$ comes equipped with a basis of $S / \mathbb{C}$, and a corresponding dual basis of $\tilde{S}$ which we denote by $\{ \beta_1^*, \ldots, \beta_5^* \}$.  Then:
\begin{equation}
8 \, Disc(A) \cdot Tr( (y_1 \beta_1^* + \ldots + y_5 \beta_5^*)^3 ) = - 3 \, F_A(y)
\end{equation}
\end{theorem}

\begin{proof}
The $G = GL_4(\mathbb{C}) \times GL_5(\mathbb{C})$ representation $\mathbb{C}^4 \otimes \wedge^2 \mathbb{C}^5$ is a prehomogeneous vector space, meaning that is has a dense open orbit.  This orbit is comprised of the elements of non-zero discriminant.  So, if we can prove that the equation is $G$-invariant and that it holds for some $A$ of non-zero discriminant, then it will hold for all $A$ of non-zero discriminant, as desired.

Taking $A = \Phi(f)$ for any binary quintic of non-zero discriminant, our knowledge of the structure coefficients of $S(A)$ enables us to prove explicitly that this formula holds.

Now to prove $G$-invariance.  Let $(g,h) \in G$ and $A' = (g,h) \cdot A$.  We know from Lemma \ref{segre action} that $F_{A'}(y) = \det(g) \det(h) \, F_A(h^t y)$.  We will prove that the left hand side of the equation transforms in the same way, by looking at the actions of $SL_4(\mathbb{C}) \times SL_5(\mathbb{C})$ and the scalar matrices $(\lambda I_4,\mu I_5)$ separately.

Let $(g,h) \in SL_4(\mathbb{C}) \times SL_5(\mathbb{C})$.  From our note above on the action of $SL_4(\mathbb{C}) \times SL_5(\mathbb{C})$ on the basis of $S(A)/\mathbb{C}$, we know that the action of $(g,h)$ on the basis $\{\beta_1^*,\ldots,\beta_5^*\}$ of $\tilde{S(A)}$ is just by $h$.  Transferring this change of basis to a transformation on $y$ amounts to replacing $y$ by $h^t y$.  The polynomial $Disc(A)$ is invariant under $SL_4(\mathbb{C}) \times SL_5(\mathbb{C})$.  Thus, both sides of the equation transform in the same way: by applying $h^t$ to $y$.

Consider the action of $(\lambda I_4,\mu I_5)$ on the equation.  The structure coefficients of the sextic resolvent algebra are degree 12 polynomials in the entries of $A$.  This means that $\lambda I_4$ acts on them by $\lambda^{12}$ and $\mu I_5$ acts by $\mu^{24}$.  Hence, $(\lambda I_4, \mu I_5)$ act on the basis of the sextic resolvent algebra by $\lambda^{12} \mu^{24}$, and so the action on the dual basis is by $\lambda^{-12} \mu^{-24}$.  The polynomial $Disc(A)$ is degree 40 and so $(\lambda,\mu)$ acts by $\lambda^{40} \mu^{80}$.  Thus $(\lambda,\mu)$ acts on the left hand side of the equation by $\lambda^4 \mu^8$.  The action on the right hand side results in $\det(\lambda I_4) \det (\mu I_5) F_A(\mu y) = \lambda^4 \mu^8 F_A(y)$ because $F_A$ is cubic, so we have equality.
\end{proof}

Let's compute the equation of the Segre cubic associated to $\Phi(f)$, which we denote by $F_f$:
\begin{IEEEeqnarray}{rCl}
F_f(y) & = & - f_0 f_2 y_2^3 - f_0 f_3 y_2^2 y_3 - f_0 f_4 y_2 y_3^2 + f_0 f_5 y_2 y_3 y_4 \nonumber \\
&& - f_0 f_5 y_3^3 + f_0 y_2^2 y_5 - f_1 f_3 y_2^2 y_4 - f_1 f_4 y_2 y_3 y_4 \nonumber \\
&& - f_1 f_5 y_3^2 y_4 - f_1 y_1 y_2^2 - f_2 f_4 y_2 y_4^2 - f_2 f_5 y_3 y_4^2 \nonumber \\
&& + f_2 y_2 y_4 y_5 - f_3 f_5 y_4^3 - f_3 y_1 y_2 y_4 + f_4 y_4^2 y_5 \nonumber \\
&& - f_5 y_1 y_4^2 - y_1^2 y_2 - y_1 y_3 y_5 - y_4 y_5^2
\end{IEEEeqnarray}

\begin{corollary} \label{segre basis property}
Let $A \in \mathbb{Z}^4 \otimes \wedge^2 \mathbb{Z}^5$.  Then $S(A)$ is dual-duogenic if and only if $F_A$ takes the following form:
\begin{equation}
F_A(y) = - y_1^2 y_2 - y_1 y_3 y_5 - y_4 y_5^2 + \text{(lower order terms in $y_1, y_5$)}
\end{equation}
\end{corollary}
\begin{proof}
The product $\beta_i^* \beta_j^* \in S \otimes \mathbb{Q}$ has an expansion in terms of the basis $\{1,\beta_1, \ldots,\beta_5\}$ of $S \otimes \mathbb{Q}$.  The $\beta_k$ coefficient is given by $Tr( \beta_i^* \beta_j^* \beta_k^*)$.  From Theorem \ref{trace cubed}, we know that these expressions are encoded in $F_A$, and the expansions of $(\beta_1^*)^2, 2 \beta_1^* \beta_5^*, (\beta_5^*)^2$ will be given by the terms of $F_A$ which are quadratic or cubic in $y_1, y_5$. 
\end{proof}

\begin{lemma} \label{matrix vanish}
Suppose $A \in \mathbb{C}^4 \otimes \wedge^2 \mathbb{C}^5$ vanishes on a 2-dimensional subspace $V \subseteq \mathbb{C}^5$.  Then $F_A(y) = 0$ for all $y \in V$.
\end{lemma}
\begin{proof}
Fix non-zero $y \in V$.  Consider the following $5 \times 5$ matrix:
\begin{equation}
\begin{pmatrix}
| & | & | & | & | \\
A_1 \, y & A_2 \, y & A_3 \, y & A_4 \, y & \bar{y}  \\
| & | & | & | & |
\end{pmatrix}^t
\end{equation}
where $\bar{y}$ denotes complex conjugation.

Its kernel contains $V \cap \{ \bar{y} \}^\perp$, hence is non-trivial.  So the determinant $F_A(y) \langle y,\bar{y} \rangle$ is zero, so $F_A(y) = 0$.
\end{proof}

We can now say exactly which quartic ring / sextic resolvent pairs are described by binary quintic forms.

First, though, we recall the subgroup $H \leqslant SL_5(\mathbb{Z})$ of all matrices of the form:

\begin{equation}
\begin{pmatrix}
1 & 0 & 0 & 0 & 0 \\
* & 1 & 0 & 0 & * \\
* & 0 & 1 & 0 & * \\
* & 0 & 0 & 1 & * \\
0 & 0 & 0 & 0 & 1
\end{pmatrix}
\end{equation}

\begin{theorem} \label{main theorem}
Let $A \in \mathbb{Z}^4 \otimes \wedge^2 \mathbb{Z}^5$.  $A$ is an $SL_4(\mathbb{Z}) \times H$ translate of $\Phi(f)$ for some integral binary quintic form $f$ if and only if the following two conditions hold:
\begin{itemize}
\item $A(e_1,e_5) = 0$
\item $S(A)$ is dual-duogenic
\end{itemize}
In this case, there is a basis of $R(A)$ arising from the binary quintic $f$.
\end{theorem}

\begin{proof}

The forward implication is covered by Theorem \ref{sextic structure2}, because $SL_4(\mathbb{Z})$ does not change basis of $S(A)$ and $H$ does not disturb either of the two properties.

The reverse implication, however, needs proof:

From $A(e_1,e_5) = 0$, we know that $A$ is of the form:
\begin{equation}
\begin{pmatrix}
0 & a & b & c & 0 \\
- a & 0 & * & * & - d \\
- b & * & 0 & * & - e \\
- c & * & * & 0 & - f \\
0 & d & e & f & 0 \\
\end{pmatrix}
\end{equation}
for some $a,b,c,d,e,f \in \mathbb{Z}[t_1,t_2,t_3,t_4]$.  The entries marked $*$ will be unimportant.

By Corollary \ref{segre basis property}, we know that the terms of $F_A$ cubic in $y_1, y_5$ all vanish and that the partial derivatives of $F_A$ evaluated at $y_1 e_1 + y_5 e_5$ take the form:
\begin{equation}
\nabla F_A (y_1 e_1 + y_5 e_5) = (0, - y_1^2, - y_1 y_5, - y_5^2, 0)
\end{equation}
These properties will impose conditions on $a,b,c,d,e,f$ which will lead us to see that we are $SL_4(\mathbb{Z})$ equivalent to some $\Phi(f)$.

First, note that by Lemma \ref{matrix vanish}, the vanishing of terms of $F_A$ which are cubic in $y_1, y_5$ is already apparent.  So, the only interesting conditions are on $\nabla F_A$.

Recall the formula $F_A(y) \langle y, z \rangle = A_1 y \wedge A_2 y \wedge A_3 y \wedge A_4 y \wedge z$.  Differentiating with respect to $y_i$, we get:
\begin{IEEEeqnarray}{rCl}
\frac{\partial F_A}{\partial y_i}(y) \langle y, z \rangle + F_A(y) z_i & = & A_1 e_i \wedge A_2 y \wedge A_3 y \wedge A_4 y \wedge z + \ldots \nonumber \\
&& + A_1 y \wedge A_2 y \wedge A_3 y \wedge A_4 e_i \wedge z
\end{IEEEeqnarray}
Taking $y = y_1 e_1 + y_5 e_5$, this simplifies to:
\begin{IEEEeqnarray}{rCl}
\frac{\partial F_A}{\partial y_i}(y) \langle y, z \rangle & = & A_1 e_i \wedge A_2 y \wedge A_3 y \wedge A_4 y \wedge z + \ldots \nonumber \\
&& + A_1 y \wedge A_2 y \wedge A_3 y \wedge A_4 e_i \wedge z
\end{IEEEeqnarray}
We can explicitly compute the determinants on the right hand side here, and thus the partial derivatives.  The partial derivatives with respect to $y_1$ and $y_5$ come out as identically 0, as we expect from Lemma \ref{matrix vanish}.  The other partial derivatives are more interesting and are listed below.  We use the notation $\Delta(pqrs)$ to mean the determinant of the $4 \times 4$ matrix whose rows are given by $p, q, r, s \in \mathbb{Z}^4 \simeq \mathbb{Z}[t_1,t_2,t_3,t_4]$.
\begin{IEEEeqnarray}{rCl}
\frac{\partial F_A}{\partial y_2}(y) & = & - y_1^2 \, \Delta(abcd) + y_1 y_5 \,  [ \Delta(abdf) - \Delta(acde) ] - y_5^2 \, \Delta(adef) \\
\frac{\partial F_A}{\partial y_3}(y) & = & - y_1^2 \, \Delta(abce) + y_1 y_5 \,  [ \Delta(abef) - \Delta(bcde) ] - y_5^2 \, \Delta(bdef) \\
\frac{\partial F_A}{\partial y_4}(y) & = & - y_1^2 \, \Delta(abcf) + y_1 y_5 \,  [ \Delta(acef) - \Delta(bcdf) ] - y_5^2 \, \Delta(cdef) 
\end{IEEEeqnarray}
Hence, we need the following equations to be satisfied by $a,b,c,d$:
\begin{IEEEeqnarray}{rCl}
\Delta(abcd) & = & 1 \\
\Delta(abdf) - \Delta(acde) & = & 0 \\
\Delta(adef) & = & 0 \\
\Delta(abce) & = & 0 \\
\Delta(abef) - \Delta(bcde) & = & -1 \\
\Delta(bdef) & = & 0 \\
\Delta(abcf) & = & 0 \\
\Delta(acef) - \Delta(bcdf) & = & 0 \\
\Delta(cdef) & = & 1
\end{IEEEeqnarray}
It is not too hard to solve these by hand, but running them through any computer algebra package will effortlessly inform you that they amount to $e = - a, f = - b, \Delta(abcd) = 1$.  This is true over any field, not just over $\mathbb{Z}$.

Thus, the matrix with rows $a,b,c,d$ lies in $SL_4(\mathbb{Z})$, and furthermore the transformation $\tau$ taking $(a,b,c,d)$ to $(t_3, - t_2, t_1, t_4)$ also lies in $SL_4(\mathbb{Z})$.  Thus, we can write:
\begin{equation} A' = \tau \cdot A = 
\begin{pmatrix}
0 & t_3 & - t_2 & t_1 & 0 \\
- t_3 & 0 & * & * & - t_4 \\
t_2 & * & 0 & * & t_3 \\
- t_1 & * & * & 0 & - t_2 \\
0 & t_4 & - t_3 & t_2 & 0 \\
\end{pmatrix}
\end{equation}

Finally, we need to find $h \in H$ to transform $A'$ into a matrix of the form $\Phi(f)$.  We just have to find $h$ which kills certain $t_i$ coefficients in the central $3 \times 3$ block of $A'$.  For example, in order to kill the $t_4$ term in $A'(e_2, e_3)$, we translate $e_3$ by a choice multiple $n$ of $e_5$, as $A'(e_2, e_3 + n e_5) = A'(e_2, e_3) - n t_4$.  These conditions amount to six linear equations in the entries of $h$, easily seen to have a unique solution; if we didn't have $A'(e_1,e_5) = 0$, these equations would be quadratic and their solubility would be unclear.
\end{proof}

\subsection{The action of $GL_2(\mathbb{Z})$ on the sextic resolvent ring}

Recall Theorem \ref{GL2}:

\begin{theorem} \nonumber
Let $f \in (Sym^5 \mathbb{Z}^2)^*$ and $\gamma \in GL_2(\mathbb{Z})$.  Then there exists $h \in H$ such that $\Phi( \gamma \cdot f) = (1,h) \cdot \sigma ( \gamma ) \cdot \Phi(f)$
\end{theorem}

This results in the following action on the basis of the sextic resolvent ring:
\begin{corollary} \label{GL2 ring}
Let $f \in (Sym^5 \mathbb{Z}^2)^*$, $\gamma \in GL_2(\mathbb{Z})$ and $g = \gamma \cdot f$.

Denote the basis of $S_f$ by $\{ 1, \beta_{1,f},\ldots,\beta_{5,f} \}$, with the usual notation $\beta_{i,f}^*$ for the dual basis of $\tilde{S_f}$. Denote the basis elements of $S_g$ and $\tilde{S_g}$ analogously.

Then:
\begin{IEEEeqnarray}{rCl}
\begin{pmatrix}
\beta_{1,g}^* \\ \beta_{5,g}^*
\end{pmatrix} & = &
\begin{pmatrix} d & - c \\ - b & a \end{pmatrix}
\begin{pmatrix}
\beta_{1,f}^* \\ \beta_{5,f}^*
\end{pmatrix} \\
\begin{pmatrix}
\beta_{2,g} \\ \beta_{3,g} \\ \beta_{4,g}
\end{pmatrix} & \equiv &
\begin{pmatrix} d^2 & - c d & c^2  \\ - 2 b d & b c + a d & - 2 a c \\ b^2 &- a b & a^2 \end{pmatrix}
\begin{pmatrix}
\beta_{2,f} \\ \beta_{3,f} \\ \beta_{4,f}
\end{pmatrix} \quad \text{mod} \quad \mathbb{Z}
\end{IEEEeqnarray}
\end{corollary}

\begin{proof}
The $GL_5$ component of $(1,h) \cdot \sigma ( \gamma )$ is of the form:
\begin{equation}
\begin{pmatrix}
d & 0 & 0 & 0 & - c \\
* & a^2 & 2ab & b^2 & * \\
* & ac & bc+ad & bd & * \\
* & c^2 & 2cd & d^2 & * \\
- b & 0 & 0 & 0 & a
\end{pmatrix}
\end{equation}

An element $A \in \mathbb{Z}^4 \otimes \wedge^2 \mathbb{Z}^5$ represents a map $\wedge^2 \tilde{S} \to \tilde{R}$, and so the action of $GL_5$ on this space (almost) corresponds to a change of basis of $\tilde{S}$, instead of $S$.  The reason we write almost is because an orientation condition on the pair $(R(A),S(A))$ means that $\tau \in GL_4(\mathbb{Z})$ will further change the basis of $S(A)$ by a factor of $\det \tau$.

Since the $GL_4$ component of $(1,h) \cdot \sigma ( \gamma )$ lies in $SL_4(\mathbb{Z})$, the $GL_5$ component represents the change of basis of $\tilde{S}$ on the nose.  The matrix above tells us the relation between the bases of $\tilde{S_f}$ and $\tilde{S_g}$, while the transpose inverse tells us how the bases of $S_f / \mathbb{Z}$ and $S_g / \mathbb{Z}$ are related.  The result follows.
\end{proof}

Note that this respects the dual-duogenicity relation between $\{ \beta_1^*, \beta_5^* \}$ and $\{ \beta_2, \beta_3, \beta_4 \}$.

There is also a subtle point to be made here: We can view $S_f^* \otimes \mathbb{Q}$ as a 6-dimensional representation of $GL_2(\mathbb{Z})$, and likewise for $S_f \otimes \mathbb{Q}$.  From the block matrix form of $\rho (\gamma) \in GL_5(\mathbb{Z})$, we see that both these representations break into irreducible components of dimensions 1, 2 and 3.  We have an equality $S_f^* \otimes \mathbb{Q} = S_f \otimes \mathbb{Q}$ given by the trace pairing, and we might then think that the subrepresentations occuring in each representation are equal.  However, this is not the case; the key point to be noted is that, on this shared space, the two $GL_2(\mathbb{Z})$-actions are in general different, as explained by the following lemma:

\begin{lemma}
Let $K$ be a field.  Suppose that $V$ is a $K$-vector space with basis $\{v_0,v_1,\ldots,v_{n-1}\}$, and with a non-degenerate pairing $\langle -,- \rangle : V \otimes_K V \to K$.  Using $\langle -,- \rangle$, we can identify $V^*$ and $V$; denote the dual basis by $\{v_0^*,v_1^*,\ldots,v_{n-1}^*\} \subset V$.  Let $\sigma \in GL(V)$ such that $\sigma(v_i) = w_i$ and $\sigma(v_i^*) = w_i^*$, where $\{w_0^*,w_1^*,\ldots,w_{n-1}^*\}$ is the dual basis of $\{w_0,w_1,\ldots,w_{n-1}\}$.  Then $\sigma$ stabilises $\langle -,- \rangle$.
\end{lemma}

\begin{proof}
The dual basis is obtained as follows:
\begin{equation}
\begin{pmatrix}
v_0^* \\ v_1^* \\ \ldots \\ v_{n-1}^*
\end{pmatrix}
=
\begin{pmatrix}
\langle v_i,v_j \rangle
\end{pmatrix}^{-1}_{i,j}
\begin{pmatrix}
v_0 \\ v_1 \\ \ldots \\ v_{n-1}
\end{pmatrix}
\end{equation}

Since $\sigma$ is $K$-linear, we have:
\begin{equation}
\begin{pmatrix}
\sigma(v_0^*) \\ \sigma(v_1^*) \\ \ldots \\ \sigma(v_{n-1}^*)
\end{pmatrix}
=
\begin{pmatrix}
\langle v_i,v_j \rangle
\end{pmatrix}^{-1}_{i,j}
\begin{pmatrix}
\sigma(v_0) \\ \sigma(v_1) \\ \ldots \\ \sigma(v_{n-1})
\end{pmatrix}
\end{equation}

But if $\sigma(v_i) = w_i$ and $\sigma(v_i^*) = w_i^*$ then this simplifies:
\begin{equation}
\begin{pmatrix}
w_0^* \\ w_1^* \\ \ldots \\ w_{n-1}^*
\end{pmatrix}
=
\begin{pmatrix}
\langle v_i,v_j \rangle
\end{pmatrix}^{-1}_{i,j}
\begin{pmatrix}
w_0 \\ w_1 \\ \ldots \\ w_{n-1}
\end{pmatrix}
\end{equation}

However, we also know to obtain $w_i^*$ in the following way:
\begin{equation}
\begin{pmatrix}
w_0^* \\ w_1^* \\ \ldots \\ w_{n-1}^*
\end{pmatrix}
=
\begin{pmatrix}
\langle w_i,w_j \rangle
\end{pmatrix}^{-1}_{i,j}
\begin{pmatrix}
w_0 \\ w_1 \\ \ldots \\ w_{n-1}
\end{pmatrix}
\end{equation}

Hence, $\langle w_i,w_j \rangle = \langle v_i,v_j \rangle$ for all $i,j$, and so $\sigma$ stabilises the pairing.
\end{proof}

This means that we should be careful when using this representation-theoretic perspective to draw links between $S_f$ and $S_f^*$.

\subsection{Classes of binary quintic forms and sextic resolvent rings}

Now that we understand how $GL_2(\mathbb{Z})$ acts on the bases of the quintic ring and sextic resolvent, we will be able to fully understand classes of binary quintics through the lens of these rings.

First, we introduce two definitions which recognise the isomorphisms between rings coming from the $GL_2(\mathbb{Z})$ action:
\begin{definition}
Let $(R,S)$ be a quintic ring / sextic resolvent pair, with both rings of non-zero discriminant.

Consider the map which sends a dual-duogenic basis $\mathfrak{B}$ of $S$ to the submodule $\mathbb{Z} \{ 1, \beta_2, \beta_3, \beta_4 \} \subseteq S$.  A \emph{dual-duogenization} of $S$ is a non-empty preimage under this map, i.e. a class of dual-duogenic bases of $S$, all producing the same sub-module via the construction $\mathbb{Z} \{ 1, \beta_2, \beta_3, \beta_4 \}$.

Equivalently, each class is determined by $\mathbb{Z} \{ \beta_1^*, \beta_5^* \}$, as this is the submodule of $S^*$ which is orthogonal to $\mathbb{Z} \{ 1, \beta_2, \beta_3, \beta_4 \}$.
\end{definition}

The key structure of a dual-duogenic based ring comes from the relation between $\{ \beta_1^*, \beta_5^* \}$ and $\{ \beta_2, \beta_3, \beta_4 \}$.  However, applying an element of $GL_2(\mathbb{Z})$ to $\{ \beta_1^*, \beta_5^* \}$ gives rise to a new dual-duogenic basis, so the key object is really $\mathbb{Z} \{ \beta_1^*, \beta_5^* \}$.  The definition of dual-duogenization is cooked up to recognise this.

\begin{definition}
Let $S$ and $S'$ be based sextic rings of non-zero discriminant, with dual-duogenic bases.  A \emph{dual-duogenic isomorphism} between $S$ and $S'$ is an isomorphism which respects the dual-duogenizations of the two rings.
\end{definition}

\begin{corollary} \label{GL2 dual duogenic}
Let $f$ be a binary quintic form of non-zero discriminant, let $\gamma \in GL_2(\mathbb{Z})$ and $g = \gamma \cdot f$.  Then 
there is a dual-duogenic isomorphism from $S_f$ to $S_g$.
\end{corollary}
\begin{proof}
Theorem \ref{GL2} implies that $S_f$ and $S_g$ are isomorphic.  Corollary \ref{GL2 ring} details this isomorphism and makes clear that it preserves the dual-duogenizations of $S_f$ and $S_g$.
\end{proof}

We can now state our main theorem relating classes of binary quintic forms to quintic rings and sextic resolvent rings:
\begin{theorem}
There is a bijection between the following two sets:
\begin{equation}
GL_2(\mathbb{Z}) \backslash (Sym^5 \mathbb{Z}^2)^* \leftrightarrow
\begin{cases}
(R,S), \text{R basis-free} \\
\text{S dual-duogenic with a fixed dual-duogenization} \\
\text{with } g(\beta_1^*,\beta_5^*) = 0 
\end{cases}
 / \sim
\end{equation}
given by $f \mapsto$ the class of $(R_f, S_f)$, where the $\sim$ denotes isomorphism of $R$ and dual-duogenic isomorphism of $S$, and where $g$ is the fundamental alternating map $\wedge^2 \tilde{S} \to \tilde{R}$.
\end{theorem}

\begin{proof}
From Theorem \ref{main theorem}, we know that every binary quintic form gives rise to a sextic resolvent ring $S_f$ with a dual-duogenic basis, with $g(\beta_1^*,\beta_5^*) = 0$, and furthermore by Corollary \ref{GL2 dual duogenic} the map above descends to classes of forms.

Conversely, let $(R,S)$ be a quintic ring / sextic resolvent pair, with a fixed dual-duogenization of $S$ and $g(\beta_1^*, \beta_5^*) = 0$.  Theorem \ref{main theorem} also tells us that there are bases of $R$ and $S$ such that $(R,S)$ is given by $\Phi(f)$ for some $f$.  The change of basis of $S$ comes from $h \in H$, so it preserves the choice of dual-duogenization.  Hence, the map in the statement of this theorem is surjective.

For injectivity, suppose $(R_f,S_f)$ and $(R_g,S_g)$ are isomorphic, with a dual-duogenic isomorphism of $S_f$ and $S_g$.  Because $\mathbb{Z} \{ \beta_1^*, \beta_5^* \}$ is fixed, one component of the change of basis from $\tilde{S_f}$ to $\tilde{S_g}$ looks like:
\begin{IEEEeqnarray}{rCl}
\begin{pmatrix}
\beta_{1,g}^* \\ \beta_{5,g}^*
\end{pmatrix} & = &
\begin{pmatrix} d & - c \\ - b & a \end{pmatrix}
\begin{pmatrix}
\beta_{1,f}^* \\ \beta_{5,f}^*
\end{pmatrix}
\end{IEEEeqnarray}
for some matrix $\begin{pmatrix} d & - c \\ - b & a \end{pmatrix} \in GL_2(\mathbb{Z})$.

Recall from Lemma \ref{recovery} that $Tr ( (u \beta_{5,f}^* - v \beta_{1,f}^*)^5) = - 10 f(u,v)$, and similarly for $g$.  It follows that $f$ and $g$ are $GL_2(\mathbb{Z})$-equivalent, as desired.
\end{proof}

\subsection{The Cayley-Klein resolvent map}

Recall from Higher Composition Laws IV the Cayley-Klein resolvent map $\psi: R \to \tilde{S}$, defined as
\begin{IEEEeqnarray}{rCll}
\alpha & \mapsto & \frac{1}{\sqrt{Disc(R)}} & (\alpha^{(1)} \alpha^{(2)} + \alpha^{(2)} \alpha^{(3)} + \alpha^{(3)} \alpha^{(4)} + \alpha^{(4)} \alpha^{(5)} + \alpha^{(5)} \alpha^{(1)} \nonumber \\
&&& - \alpha^{(1)} \alpha^{(3)} - \alpha^{(3)} \alpha^{(5)} - \alpha^{(5)} \alpha^{(2)} - \alpha^{(2)} \alpha^{(4)} - \alpha^{(4)} \alpha^{(1)}) \nonumber \\
\end{IEEEeqnarray}

\begin{lemma}
The Cayley-Klein map $\psi : R_f \to \tilde{S_f}$ has the following property:
\begin{equation}
\psi(x^3 \alpha_1 + x^2 y \alpha_2 + x y^2 \alpha_3 + y^3 \alpha_4) = 4 f(x,y) ( y \beta_1^* - x \beta_5^*)
\end{equation}
\end{lemma}

\begin{proof}
Bhargava shows that, for $A \in \mathbb{Z}^4 \otimes \wedge^2 \mathbb{Z}^5$ and the associated based quintic ring $R = R(A)$, this map is given by the $4 \times 4$ sub-Pfaffians $(P_1,\ldots,P_5)$ of $A$ as follows:
\begin{equation}
\psi(t_1 \alpha_1 + \ldots + t_4 \alpha_4) = 4 P_1(t) \beta_1^* + \ldots + 4 P_5(t) \beta_5^*
\end{equation}

[Q: Are the sub-pfaffians signed here?]

Recall that when $A = \Phi(f)$, the sub-pfaffians are in the span of the following five quadrics:
\begin{IEEEeqnarray}{rCl}
Q_1 & = & t_1 t_3 - t_2^2\\
Q_2 & = & t_1 t_4 - t_2 t_3\\
Q_3 & = & t_2 t_4 - t_3^2\\
Q_4 & = &  f_0 t_1 t_2 +  f_1 t_2^2 +  f_2 t_2 t_3 +  f_3 t_3^2 +  f_4 t_3 t_4 +  f_5 t_4^2\\
Q_5 & = &  f_0 t_1^2 +  f_1 t_1 t_2 +  f_2 t_2^2 +  f_3 t_2 t_3 +  f_4 t_3^2 +  f_5 t_3 t_4
\end{IEEEeqnarray}

In fact, the five $4 \times 4$ signed sub-pfaffians are easily seen to be:
\begin{equation}
Q(A) = [Q_4, Q_1, Q_2, Q_3, - Q_5]
\end{equation}

When $A = \Phi(f)$ and $(t_1,\ldots,t_4) = (x^3, x^2 y, x y^2, y^3)$, because of the special form of the sub-Pfaffians, they simplify to give the stated result.
\end{proof}

This mimics the behaviour of the resolvent maps for binary cubics and quartics:
\begin{itemize}
\item For a binary cubic, the resolvent map has the property:
\begin{equation}
x \alpha_1 + y \alpha_2 \mapsto f(x,y) \omega
\end{equation}
where $\omega$ generates the quadratic resolvent ring.

\item For a binary quartic, the resolvent map has the property:
\begin{equation}
x^2 \alpha_1 + x y \alpha_2 + y^2 \alpha_3 \mapsto f(x,y) \omega
\end{equation}
where $\omega$ generates the cubic resolvent ring.
\end{itemize}

So, it seems that the resolvent map picks out some of the key structure of the resolvent ring, when evaluated on a certain family of elements corresponding to the relevant rational normal curve.

\section{Counting binary quintics of bounded discriminant}

[To be added.]

\end{document}