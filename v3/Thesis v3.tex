\documentclass{report}
\usepackage{amsmath}
\usepackage{amsthm}
\usepackage{amssymb}
\usepackage[retainorgcmds]{IEEEtrantools}
\usepackage{tikz-cd}

\begin{document}
\title{Binary Forms \& Rings of Low Rank}
\author{Dan Fess}
\date{10th March 2021}
\maketitle

\newtheorem{theorem}{Theorem}
\newtheorem{lemma}{Lemma}[section]
\newtheorem{prop}[lemma]{Proposition}
\newtheorem{corollary}[lemma]{Corollary}
\newtheorem{conjecture}{Conjecture}
\newtheorem{definition}[lemma]{Definition}

\chapter{Introduction}

By a construction of Birch and Merriman, we can attach to a binary quintic form $f$ a quintic ring $R_f$.  Not all quintic rings are produced this way; the forms describe a certain subclass of quintic rings.  What is special about this subclass?  Which quintic rings are these?  We can also associate to a quintic ring a sextic resolvent ring; perhaps there is something special about the sextic resolvent?

Another problem we can hope to understand is the number of $GL_2(\mathbb{Z})$-classes of binary quintic forms of bounded discriminant.  The collection of all quintic rings is described by the space $\mathbb{Z}^4 \otimes \wedge^2 \mathbb{Z}^5$, which naturally has an action of $\Gamma = GL_4(\mathbb{Z}) \times GL_5(\mathbb{Z})$.  If we can find a discriminant-preserving, orbit-preserving map from binary quintic forms to this space, then we can hope to translate our knowledge of orbits in $\mathbb{Z}^4 \otimes \wedge^2 \mathbb{Z}^5$ to study classes of binary quintics.

\section{Results}

Denote a generic element $f \in (Sym^5 \mathbb{Z}^2)^*$ by $f = f_0 x^5 + f_1 x^4 y + f_2 x^3 y^2 + f_3 x^2 y^3 + f_4 x y^4 + f_5 y^5$.

\begin{theorem}
There is a map $\Phi : (Sym^5 \mathbb{Z}^2)^* \to \mathbb{Z}^4 \otimes \wedge^2 \mathbb{Z}^5$ which respects the two constructions of quintic rings.  Furthermore, it is discriminant- and orbit-preserving.  Explictly:\\

$\Phi(f) = \begin{pmatrix}
0 & t_3 & - t_2 & t_1 & 0\\
- t_3 & 0 & -  f_0 t_1 -  f_1 t_2 & -  f_2 t_2 -  f_3 t_3 & - t_4\\
t_2 &  f_0 t_1 +  f_1 t_2 & 0 & -  f_4 t_3 -  f_5 t_4 & t_3\\
- t_1 &  f_2 t_2 +  f_3 t_3 &  f_4 t_3 +  f_5 t_4 & 0 & - t_2\\
0 & t_4 & - t_3 & t_2 & 0\\
\end{pmatrix}$\\
\end{theorem}

For $A \in \mathbb{Z}^4 \otimes \wedge^2 \mathbb{Z}^5$, denote the associated based quintic ring by $R(A)$ and the based sextic resolvent ring by $S(A)$.  This theorem tells us that $R_f = R(\Phi(f))$, but the more interesting ramification is that it gives us access to the sextic resolvent ring $S(\Phi(f))$, which we denote by $S_f$.  We can explicitly compute its multiplicative structure, and find out that the sextic resolvent rings coming from binary quintic forms have the following unusual structure:

\begin{definition} \label{dual digenic}
Let $(R,S)$ be a quintic ring / sextic resolvent pair, with both rings of non-zero discriminant.  Consider $S$ as a based ring, with basis denoted by $\{1,\beta_1,\ldots,\beta_5\}$.  Denote the dual basis with respect to the trace pairing by $\{\beta_0^*,\beta_1^*,\ldots,\beta_5^*\}$, i.e. $\beta_i^* \in S \otimes \mathbb{Q}$ such that $Tr(\beta_i^* \beta_j) = \delta_{ij}$.

Then, we say that the based ring $S$ is \emph{dual-digenic} if it has the following property:
\begin{equation}
(\beta_2,\beta_3,\beta_4) \equiv 8 \, Disc(R) \cdot ( (\beta_1^*)^2, 2 \beta_1^* \beta_5^*, (\beta_5^*)^2 ) \, \text{mod} \, \, \mathbb{Q}
\end{equation}

This can also be rephrased in terms of $Disc(S)$ using $Disc(S) = (16 \cdot Disc(R))^3$.

We also define a (non-based) ring $S$ to be dual-digenic if it has a basis in which it is dual-digenic.
\end{definition}

\begin{theorem} \label{sextic structure}
Let $f$ be a binary quintic form of non-zero discriminant $\Delta(f)$.  Then, the based ring $S_f$ is dual-digenic.
\end{theorem}

Nakagawa proved that if $f$ and $g$ are $GL_2(\mathbb{Z})$-equivalent binary quintic forms, their associated quintic rings $R_f$ and $R_g$ are isomorphic, but have different canonical bases; the $GL_2(\mathbb{Z})$ action induces this change of basis of the quintic ring, and can equally be viewed as a $GL_2(\mathbb{Z})$ action on $R_f / \mathbb{Z} \simeq \mathbb{Z}^4$.  Similarly, in the sextic resolvent ring, the $GL_2(\mathbb{Z})$ action induces a change of basis and a corresponding change of dual basis, which respect the special property of dual-digenicity.

\begin{theorem}
Let $f,g$ be binary quintic forms with $\gamma \in GL_2(\mathbb{Z})$ such that $g = \gamma \cdot f$.  Then, $S_f$ and $S_g$ have different canonical bases but are isomorphic in a way that respects dual-digenicity, i.e. $\gamma$ acts linearly on $sp \{\beta_1^*, \beta_5^*\}$ and quadratically on $sp \{\beta_2,\beta_3,\beta_4\}$.
\end{theorem}

We would like to exactly pin down the quintic ring / sextic resolvent pairs $(R,S)$ given by binary quintics.  Thus, we need to know if this property is sufficient for the pair to have come from a binary quintic.  With an extra condition on the alternating matrix $A$, there is indeed a basis of $R$ such that it comes from a binary quintic form:
\begin{theorem}
Let $(R,S)$ be a quintic ring / sextic resolvent pair, basis-free, of non-zero discriminant, with fundamental alternating map $g : \wedge^2 \tilde{S} \to \tilde{R}$.  Then, $(R,S)$ arise from some binary quintic form if and only if the following two conditions hold:
\begin{itemize}
\item $S$ has a basis $\{1,\beta_1,\ldots,\beta_5\}$ in which it is dual-digenic
\item The associated dual basis has the property $g(\beta_1^*,\beta_5^*) = 0$
\end{itemize}
\end{theorem}

By studying the action of various groups involved in this picture, we can upgrade this theorem so that it concerns classes of binary quintic forms.  This theorem involves the notion of a choice of dual-digenization, which will be defined later, in Section \ref{sextic section}; in short, it recognises the different ways in which a sextic ring can have a dual-digenic basis.  Similarly, we shall later define a dual-digenic isomorphism; roughly speaking, this is an isomorphism of dual-digenic sextic rings, which respects their dual-digenizations.
\begin{theorem}
There is a bijection between the following two sets:
\begin{equation}
GL_2(\mathbb{Z}) \backslash 
(Sym^5 \mathbb{Z}^2)^* \leftrightarrow
\left\{
\begin{array}{l}
(R,S), \text{R basis-free} \\
\text{S dual-digenic with a} \\
\text{fixed dual-digenization} \\
\text{and } g(\beta_1^*,\beta_5^*) = 0 
\end{array}
\right\}
 / \sim
\end{equation}
given by $f \mapsto$ the class of $(R_f, S_f)$, where the $\sim$ denotes isomorphism of $R$ and dual-digenic isomorphism of $S$, and where $g$ is the fundamental alternating map $\wedge^2 \tilde{S} \to \tilde{R}$.
\end{theorem}

The other direction in which we would like to take this work is to count classes of binary quintic forms of bounded discriminant.  The number of $\Gamma$-orbits in $\mathbb{Z}^4 \otimes \wedge^2 \mathbb{Z}^5$ of bounded discriminant is understood, and the map $\Phi$ is discriminant- and orbit-preserving, so if we understand $\Phi$ well then we can count classes of binary quintics.  Specifically, we need to know how many classes of binary quintics can possibly land in one $\Gamma$-orbit in $\mathbb{Z}^4 \otimes \wedge^2 \mathbb{Z}^5$.  The following result goes some way towards this:

\begin{theorem} [In Progress]
Given a binary quintic $f \in (Sym^5 \mathbb{Z}^2)^*$, there is an associated Segre cubic threefold $V_f \subseteq \mathbb{P}^4$.  The Fano variety of lines on $V_f$ has a distinguished 2-dimensional component $D \subseteq \mathbb{P}^9$.  There is an explicit 2-dimensional subvariety of the cone over $D$ in $\mathbb{A}^{10}$, whose integral points correspond to classes of forms which land in the same $\Gamma$-orbit as $\Phi(f)$.  In particular, this subvariety has finitely many integral points.
\end{theorem}

This theorem is proved, except for the statement that the subvariety in question is of dimension 2.   I have only proved this in the case of $f = x^5 + y^5$.  In the general case, I currently know that it is of dimension at most 2.  [Note to self: acting by $GL_4(\mathbb{C}) \times GL_5(\mathbb{C})$ should give dimension 2 in the general case.]

\chapter{Preliminaries regarding $n$-ic rings}

To add:
\begin{itemize}
\item Cubic and quartic rings ( + resolvents)
\item More detail on quintic rings / sextic resolvents
\item $n$-ic rings from binary $n$-ic forms
\item Relevant material on $n$-ic rings in general
\end{itemize}

\begin{definition}
Let $\rho_r : GL_2(\mathbb{Z}) \to GL_{r+1}(\mathbb{Z})$ be given by the standard action on degree $r$ monomials, i.e. if $f(x,y) = \sum f_i x^{r-i} y^i$ and $(\gamma \cdot f)(x,y) = \sum g_i x^{r-i} y^i$, then
\begin{equation}
\begin{pmatrix}
g_0 \\ g_1 \\ \ldots \\ g_r
\end{pmatrix}
= \rho_r (\gamma)
\begin{pmatrix}
f_0 \\ f_1 \\ \ldots \\ f_r
\end{pmatrix}
\end{equation}

More explicitly, the map looks like:
\begin{equation}
\rho_r (\gamma) =
\begin{pmatrix}
a^r & a^{r-1} b & \ldots & b^r \\
r a^{r-1} c & \ldots & \ldots & r b^{r-1} d\\
\ldots & \ldots & \ldots & \ldots \\
c^r & \ldots & \ldots & d^r
\end{pmatrix}
\end{equation}

\end{definition}


\begin{theorem} [Nakagawa, (+ Wood for $f_0$ non-zero?) ] \label{nakagawa}
Let $f$ be a binary $n$-ic form, $\gamma \in GL_2(\mathbb{Z})$ and $g = \gamma \cdot f$.  Denote the basis of $R_f$ by $\{1, \alpha_{1,f},\ldots,\alpha_{n-2,f}\}$, and denote the basis of $R_g$ analogously.

Then, there exists an isomorphism $\sigma : R_f \to R_g$, with the property:
\begin{equation}
\begin{pmatrix}
\alpha_{1,g} \\ \alpha_{2,g} \\ \ldots \\ \alpha_{n-1,g}
\end{pmatrix}
\equiv
\det(\gamma) \, \rho_{n-2} (\gamma)
\begin{pmatrix}
\sigma(\alpha_{1,f}) \\ \sigma(\alpha_{2,f}) \\ \ldots \\ \sigma(\alpha_{n-1,f})
\end{pmatrix}
\text{ mod } \mathbb{Z}
\end{equation}

\end{theorem}

We can associate to a based $n$-ic ring $R$ of non-zero discriminant a well-defined set of $n$ points $X_R \subseteq \mathbb{P}^{n-2}$ as follows:

\begin{definition} \label{X_R def}
Let $R$ have basis $\{1,\alpha_1,\ldots,\alpha_{n-1} \}$ and corresponding dual basis $\{ \alpha_0^*, \alpha_1^*, \ldots, \alpha_{n-1}^* \}$.  Denote the $n$ homomorphisms $\rho : R \otimes \mathbb{Q} \to \mathbb{C}$ by $\rho^{(1)},\ldots,\rho^{(n)}$.  Let $P_k = (\rho^{(k)}(\alpha_1^*),\ldots,\rho^{(k)}(\alpha_{n-1}^*)$ for $1 \leq k \leq n$.  Then we define $X_R = \{ P_1,\ldots,P_n \}$.
\end{definition}

Note that while the $\alpha_i$ are only canonical mod $\mathbb{Z}$, the $\alpha_i^*$ are totally canonical.  Furthermore, the lemma below shows that $P_k \neq (0,\ldots,0)$, so these genuinely define points in $\mathbb{P}^{n-2}$, and they are in general position.

\begin{lemma} \label{X_R well-defined}
$P_k \neq (0,\ldots,0)$ for any $k$, and furthermore they lie in general position in the sense that no $n-1$ of them lie on a hyperplane.
\end{lemma}
\begin{proof}
We use the shorthand $\alpha^{(k)}$ to mean $\rho^{(k)}(\alpha)$.

Define the matrix
\begin{equation}
D = \begin{pmatrix}
1 & 1 & \ldots & 1 \\
\alpha_1^{(1)} & \alpha_1^{(2)} & \ldots & \alpha_1^{(n)} \\
\ldots & & & \ldots \\
\alpha_{n-1}^{(1)} & \alpha_{n-1}^{(2)} & \ldots & \alpha_{n-1}^{(n)}
\end{pmatrix}
\end{equation}

Note that $\det(D) = Disc(R) \neq 0$, so $D$ is invertible.  In fact, $D^{-1}$ has to do with the dual basis:

\begin{equation}
(D^t)^{-1} = \begin{pmatrix}
(\alpha_0^*)^{(1)} & (\alpha_0^*)^{(2)} & \ldots & (\alpha_0^*)^{(n)} \\
(\alpha_1^*)^{(1)} & (\alpha_1^*)^{(2)} & \ldots & (\alpha_1^*)^{(n)} \\
\ldots & & & \ldots \\
(\alpha_{n-1}^*)^{(1)} & (\alpha_{n-1}^*)^{(2)} & \ldots & (\alpha_{n-1}^*)^{(n)}
\end{pmatrix}
\end{equation}

If we add all the columns to the last one, because $Tr(\alpha_i^*) = 0$ for all $i \neq 0$ and $Tr(\alpha_0^*) = 1$, we obtain the following matrix:
\begin{equation}
\begin{pmatrix}
(\alpha_0^*)^{(1)} & (\alpha_0^*)^{(2)} & \ldots & (\alpha_0^*)^{(n-1)} & 1 \\
(\alpha_1^*)^{(1)} & (\alpha_1^*)^{(2)} & \ldots & (\alpha_1^*)^{(n-1)} & 0 \\
\ldots & & & \ldots \\
(\alpha_{n-1}^*)^{(1)} & (\alpha_{n-1}^*)^{(2)} & \ldots & (\alpha_{n-1}^*)^{(n-1)} & 0
\end{pmatrix}
\end{equation}

This matrix is invertible, its determinant that of the $(1,n)$-th minor, from which we see that $\{ P_1, \ldots, P_{n-1} \}$ is linearly independent.

Applying this trick to each of the columns in turn results in the statement of the lemma.
\end{proof}



\chapter{New results on binary $n$-ic forms and $n$-ic rings}

Here we collect a few new and useful results on the based rings coming from binary $n$-ic forms.

Throughout this section, let $R = \mathbb{Z}\{1,\alpha_1,\ldots,\alpha_{n-1}\}$ be a based $n$-ic ring, denote the based $n$-ic ring coming from a binary $n$-ic form $f$ by $R_f$, and denote a generic element $\gamma \in GL_2(\mathbb{Z})$ by $\begin{pmatrix} a & b \\ c & d \end{pmatrix}$.

\section{An identity of the index form}

The form $f$ is encoded in the index form of $R_f$ in the following way:

\begin{theorem}
In the based ring $R_f$, let $\alpha_f(x,y) = \sum_{i=0}^{n-2} x^{n-2-i} y^i \alpha_i$.  Denote the index form in $R_f$ by $I_f$.  Then:
\begin{equation}
I_f(\alpha_f(x,y)) = f(x,y)^{\frac{(n-1)(n-2)}{2}}
\end{equation}
\end{theorem}

\begin{proof}
First, we claim that both sides of this equation transform in the same way under the action of $\gamma \in SL_2(\mathbb{Z})$.  Let $g = \gamma \cdot f$.  From Theorem \ref{nakagawa}, the bases of $R_f / \mathbb{Z}$ and $R_g / \mathbb{Z}$ are related by $\rho_{n-2}(\gamma) \in SL_{n-1}(\mathbb{Z})$.   Hence, $\alpha_g (x,y) = \alpha_f (ax+cy,bx+dy)$.  There is no change of orientation, so $I_f = I_g$ and both sides transform linearly in the same way under the action of $\gamma$.

Now, we shall apply a well-chosen element of $SL_2(\mathbb{Z})$ to reduce the proof to a simple case.

Suppose $(x,y)$ are coprime.  Choose an element $\gamma \in SL_2(\mathbb{Z})$ mapping $(1,0)^t$ to $(x,y)^t$.  Let $g = \gamma^t \cdot f$.  Then the first basis element of $R_g / \mathbb{Z}$ will be $\alpha_g(1,0) = \alpha_f(x,y)$ and $I_f = I_g$, i.e. no orientation change.  So, proving $I_f(\alpha_f(x,y)) = f(x,y)^{\frac{(n-1)(n-2)}{2}}$ amounts to proving $I_g(\alpha_g(1,0)) = g(1,0)^{\frac{(n-1)(n-2)}{2}}$.

When $(x,y)$ are not coprime, we can still choose an element $\gamma \in SL_2(\mathbb{Z})$ mapping $(z,0)^t$ to $(x,y)^t$ for $z = gcd(x,y)$.  In this case, proving $I_f(\alpha_f(x,y)) = f(x,y)^{\frac{(n-1)(n-2)}{2}}$ amounts to proving $I_g(\alpha_g(z,0)) = g(z,0)^{\frac{(n-1)(n-2)}{2}}$.  This is the simple case we shall prove directly:

With explicit calculation using Theorem \ref{nakagawa}, the element $\alpha_g(z,0) = z^{n-2} \alpha_1$ has $I_g( z^{n-2} \alpha_1 ) = z^\frac{n(n-1)(n-2)}{2} g_0^\frac{(n-1)(n-2)}{2} = (g_0 z^n)^\frac{(n-1)(n-2)}{2} = g(z,0)^\frac{(n-1)(n-2)}{2}$.
\end{proof}

\section{Binary $n$-ic forms and the rational normal curve}

Given an $n$-ic ring $R$ of non-zero discriminant, recall that we can associate to $R$ a set of $n$ points $X_R \subseteq \mathbb{P}^{n-2}$.

\begin{definition}
Denote the rational normal curve in $\mathbb{P}^{n-2}$ by $V_{n-2}$, i.e. $V_{n-2} = \{ ( \lambda^{n-2} : \lambda^{n-3} \mu : \ldots : \mu^{n-2} ) : ( \lambda : \mu ) \in \mathbb{P}^1 \}$.
\end{definition}

The main result we have on $n$-ic rings coming from binary $n$-ic forms is:

\begin{theorem} \label{n-ic theorem}
For $n \geq 3$, a based n-ic ring $R$ of non-zero discriminant is given by a binary $n$-ic form if and only if $X_R$ is contained in the rational normal curve $V_{n-2} \subseteq \mathbb{P}^{n-2}$.
\end{theorem}

Furthermore, we can be more precise about the set $X_{R_f}$:
\begin{definition}
Let $q_r : \mathbb{C}^2 \to \mathbb{C}^{r+1}$ be given by
\begin{equation}
\begin{pmatrix} u \\ v \end{pmatrix} \mapsto \begin{pmatrix} u^r \\ u^{r-1} \\ \ldots \\ v^r \end{pmatrix}
\end{equation}
so that $q_r(u,v)$ is a generic element of $V_r$.
\end{definition}

\begin{theorem}  \label{X_Rf}
Let $f$ be a binary $n$-ic form of non-zero discriminant.  Then $X_{R_f} = \{ q_{n-2}(\lambda , \mu) : f( \lambda, \mu ) = 0 \}$, and in particular $X_{R_f} \subseteq V_{n-2}$.
\end{theorem}

We will first prove Theorem \ref{X_Rf} and thus the forward direction of Theorem \ref{n-ic theorem}.  The converse of Theorem \ref{n-ic theorem} will require more work.  Our plan of attack will be to reduce these results to the special case of a ring given by a binary $n$-ic form $f$ with $f_0 \neq 0$, because in this case we have an explicit basis of $R_f$.  We will use $SL_2(\mathbb{Z})$ to make this reduction.

We set to work proving Theorem \ref{X_Rf} in this special case:

\begin{lemma} \label{points f0}
Let $f$ be a binary $n$-ic form of non-zero discriminant with $f_0 \neq 0$.  Then $X_{R_f} = \{ q_{n-2}(\lambda , \mu) : f( \lambda, \mu ) = 0 \}$.  In particular, $X_{R_f} \subseteq U_{n-2} := \{ (x_0, x_1, \ldots, x_{n-2} : x_{n-2} \neq 0 \} \subseteq \mathbb{P}^{n-2}$.
\end{lemma}
\begin{proof}
$R$ can be described as the subring of $\mathbb{Q}[x]/(f(x,1))$ with the following basis, where $\theta$ is the image of $x$:
\begin{IEEEeqnarray}{rCl}
\alpha_0 & = & 1\\
\alpha_1 & = & f_0 \theta\\
\alpha_2 & = & f_0 \theta^2 + f_1 \theta \\
& \ldots & \nonumber \\
\alpha_{n-1} & = & f_0 \theta^{n-1} + f_1 \theta^{n-2} + \ldots + f_{n-2} \theta
\end{IEEEeqnarray}
We now prove that $\alpha_j^* = \alpha_{j+1}^* \theta$ for $1 \leq j \leq n-2$.  First, note that $Tr(\alpha_0 \cdot \alpha_{j+1}^* \theta) = Tr(\frac{1}{f_0} \alpha_1 \cdot \alpha_{j+1}^*) = 0$ for $j \neq 0$.    Now, for $1 \leq i,j \leq n-2$, note that:
\begin{IEEEeqnarray}{rCl}
Tr(\alpha_i \cdot \alpha_{j+1}^* \theta) & = & Tr(\theta \alpha_i \cdot \alpha_{j+1}^*) \\
& = & Tr((\alpha_{i+1} - f_i \theta) \cdot \alpha_{j+1}^*) \\
& = & \delta_{ij}
\end{IEEEeqnarray}
Hence, $\alpha_j^* = \alpha_{j+1}^* \theta$ for $1 \leq j \leq n-2$, so $(\alpha_1^* : \ldots : \alpha_{n-1}^*) = (\theta^{n-2} \alpha_{n-1}^* : \theta^{n-3} \alpha_{n-1}^* : \ldots : \alpha_{n-1}^*)$, and $\rho^{(k)}(\alpha_1^* : \ldots : \alpha_{n-1}^*) = (\theta_k^{n-2} : \theta_k^{n-3} : \ldots : 1) \in V_{n-2}$, where $\theta_1, \ldots, \theta_n$ are the roots of $f(x,1) = 0$.  (Note that $\rho^{(k)}(\alpha_{n-1}^*) \neq 0$ for all $k$; otherwise, for some $k$, $\rho^{(k)}(\alpha_j^*) = 0$ for all $j$, but this is outlawed by Lemma \ref{X_R well-defined}.)
\end{proof}

We will use $SL_2(\mathbb{Z})$ to reduce to this special case, which means that we need to understand how $X_{R_f}$ and $X_{R_g}$ are related for equivalent forms $f,g$.
\begin{lemma}
Let $\gamma \in SL_2(\mathbb{Z})$, $f$ be a binary $n$-ic form of non-zero discriminant and $g = \gamma \cdot f$.  Then $X_{R_f} = \rho_{n-2}(\gamma)^t X_{R_g}$.
\end{lemma}
\begin{proof}
By Theorem \ref{nakagawa}, to get from the basis of $R_g$ to the basis of $R_f$, we apply $\rho_{n-2}(\gamma^{-1})$.  Hence, the relation between the dual bases is given by the transpose inverse of this matrix, namely $\rho_{n-2}(\gamma)^t$.  The sets $X_{R_f}$ and $X_{R_g}$ come from the dual bases of the rings, and the lemma follows.
\end{proof}

\begin{lemma} \label{stabiliser}
$\rho_r(\gamma)^t q_r(u,v) = q_r((u,v) \gamma )$, and so $\rho_r(\gamma)^t$ stabilises $V_r$.
\end{lemma}
\begin{proof}
From the definition of $\rho_r(\gamma)$:
\begin{equation}
\begin{pmatrix} g_0 \\ g_1 \\ \ldots \\ g_r \end{pmatrix}
= 
\rho_r(\gamma)
\begin{pmatrix} f_0 \\ f_1 \\ \ldots \\ f_r \end{pmatrix}
\end{equation}

We calculate:
\begin{IEEEeqnarray}{rCl}
q_r(u,v)^t \rho_r(\gamma)
\begin{pmatrix} f_0 \\ f_1 \\ \ldots \\ f_r \end{pmatrix}
& = & q_r(u,v)^t
\begin{pmatrix} g_0 \\ g_1 \\ \ldots \\ g_r \end{pmatrix}\\
& = & g(u,v) \\
& = & f( (u,v) \gamma ) \\
& = & q_r( (u,v) \gamma )^t
\begin{pmatrix} f_0 \\ f_1 \\ \ldots \\ f_r \end{pmatrix}
\end{IEEEeqnarray}

This holds identically in the $f_i$, so
\begin{equation}
\rho_r(\gamma)^t q_r(u,v) = q_r( (u,v) \gamma )
\end{equation}
\end{proof}

\begin{proof} [Proof of Theorem \ref{X_Rf}]
The form $f \neq 0$, so there exists $\gamma \in SL_2(\mathbb{Z})$ such that $(\gamma \cdot f)(1,0) \neq 0$.  Let $g = \gamma \cdot f$.  Then by Lemma \ref{points f0}, $X_{R_g}$ is of the required form, and by Lemma \ref{stabiliser}:
\begin{IEEEeqnarray}{rCl}
X_{R_f} & = & \rho_{n-2}(\gamma)^t X_{R_g} \\
& = & \rho_{n-2}(\gamma)^t \cdot \{ q_{n-2}(\lambda , \mu) : g( \lambda, \mu ) = 0 \} \\
& = & \rho_{n-2}(\gamma)^t \cdot \{ q_{n-2}(\lambda , \mu) : f( (\lambda, \mu) \gamma ) = 0 \} \\
& = & \{ q_{n-2}( (\lambda , \mu) \gamma ) : f( (\lambda, \mu) \gamma ) = 0 \} \\
& = & \{ q_{n-2}( \lambda , \mu ) : f( \lambda, \mu ) = 0 \}
\end{IEEEeqnarray}
\end{proof}

This deals with the forward direction of Theorem \ref{n-ic theorem} too.  Now, we set about proving that if $X_R \subseteq V_{n-2}$ then $R = R_f$ for some $f$.  Again, we reduce to the case $f_0 \neq 0$ first.

\begin{definition}
Let $R$ be a based $n$-ic ring with basis $\{1,\alpha_1,\ldots,\alpha_{n-1}\}$ and $\gamma \in GL_2(\mathbb{Z})$.  We define $\gamma \cdot R$ to be the based $n$-ic ring with the following basis mod $\mathbb{Z}$:

\begin{equation}
\begin{pmatrix}
\alpha_{1}' \\ \alpha_{2}' \\ \ldots \\ \alpha_{n-1}'
\end{pmatrix}
\equiv
\det(\gamma) \, \rho_{n-2} (\gamma)
\begin{pmatrix}
\alpha_{1} \\ \alpha_{2} \\ \ldots \\ \alpha_{n-1}
\end{pmatrix}
\text{ mod } \mathbb{Z}
\end{equation}

\end{definition}

\begin{corollary}
$\gamma \cdot R_f = R_{\gamma \cdot f}$
\end{corollary}
\begin{proof}
Evident from Theorem \ref{nakagawa}.
\end{proof}

\begin{corollary} \label{GL2 invariance}
The statement of Theorem \ref{n-ic theorem} is $GL_2(\mathbb{Z})$-invariant.  In other words:

(i) A based $n$-ic ring $R$ is given by a binary $n$-ic form if and only if $\gamma \cdot R$ is.

(ii) $X_R \subseteq V_{n-2}$ if and only if $X_{\gamma \cdot R} \subseteq V_{n-2}$
\end{corollary}
\begin{proof}
(i) Suppose $R = R_f$, then Theorem \ref{nakagawa} tells us that $\gamma \cdot R = R_{\gamma \cdot f}$.  The converse holds by applying $\gamma^{-1}$ and using the same argument.

(ii) The bases of $R$ and $\gamma \cdot R$ are related by $\det(\gamma) \rho_{n-2}(\gamma)$, and so their dual bases are related by $\det(\gamma) (\rho_{n-2}(\gamma)^t)^{-1}$.  Hence, the sets of points $X_R$ and $X_{\gamma \cdot R}$ in $\mathbb{P}^{n-2}$ are related by $(\rho_{n-2}(\gamma)^t)^{-1}$, which stabilises $V_{n-2}$ as proven in Lemma \ref{stabiliser}.
\end{proof}

\begin{lemma}
Suppose $X_R \subseteq V_{n-2}$.  Then there exists $\gamma \in SL_2(\mathbb{Z})$ such that $X_{\gamma \cdot R} \subseteq V_{n-2} \cap U_{n-2}$.
\end{lemma}
\begin{proof}
Let $X_R = \{ q_{n-2}(\lambda_i, \mu_i) : 1 \leq i \leq n \}$.  By Lemma \ref{stabiliser} and the proof of Corollary \ref{GL2 invariance} (ii), $X_{\gamma \cdot R} = \{ q_{n-2}((\lambda_i, \mu_i) \gamma^{-1}) : 1 \leq i \leq n \}$.  For $\gamma = \begin{pmatrix} a & b \\ c & d \end{pmatrix}$, the condition that $X_{\gamma \cdot R} \subseteq U_{n-2}$ amounts to $ - b \lambda_i + a \mu_i \neq 0$ for all $i$.  This outlaws at most 2 coprime tuples $(a,b)$ for each $(\lambda_i , \mu_i)$, leaving infinitely many $\gamma$ to choose from which will satisfy our condition.
\end{proof}

The above results enable us to reduce the proof of the backward direction of Theorem \ref{n-ic theorem} to the following result:

\begin{prop} \label{U_n-2 theorem}
Let $R$ be a based $n$-ic ring ($n \geq 3$) with $X_R \subseteq V_{n-2} \cap U_{n-2}$.  Then $R = R_f$ for some $f$ with $f_0 \neq 0$.
\end{prop}
\begin{proof}
Denote the homomorphisms $\rho : R \to \mathbb{C}$ by $\rho^{(1)},\ldots,\rho^{(n)}$.  Because of the assumption on $X_R$, we can let $X_R = \{P_1, \ldots, P_n\}$ with $P_k = (\theta_k^{n-2} : \ldots : \theta_k : 1)$.  Hence, $\rho^{(k)}(\alpha_{n-1}^*) \neq 0$ for all $k$, and so $\alpha_{n-1}^* \in (R \otimes \mathbb{Q})^*$.  We also see that $\rho^{(k)}(\alpha_j^*) = \rho^{(k)}(\alpha_{j+1}^*) \theta_k$ for $1 \leq j \leq n-2$, which in particular tells us $\rho^{(k)}(\alpha_{n-2}^*) = \rho^{(k)}(\alpha_{n-1}^*) \theta_k$.  Let $\theta = \frac{\alpha_{n-2}^*}{\alpha_{n-1}^*} \in R \otimes \mathbb{Q}$.  Then $\rho^{(k)}(\theta) = \theta_k$ for all $k$, so $\rho^{(k)}(\alpha_j^*) = \rho^{(k)}(\alpha_{j+1}^* \theta)$ for all $k$ and $1 \leq j \leq n-2$.  The $n$ homomorphisms $\rho^{(k)} : R \otimes \mathbb{Q} \to \mathbb{C}$ separate points, so we actually have $\alpha_j^* = \alpha_{j+1}^* \theta$ for $1 \leq j \leq n-2$.

With the aim of showing that $\{\alpha_1, \ldots, \alpha_{n-1}\}$ takes the special form given by a binary n-ic form, we consider $\theta \alpha_i - \alpha_{i+1}$.  For $1 \leq j \leq n-2$ and $0 \leq i \leq n-2$:
\begin{IEEEeqnarray}{rCl}
Tr((\theta \alpha_i - \alpha_{i+1}) \cdot \alpha_{j+1}^*) & = & Tr(\alpha_i \cdot \alpha_j^*) - Tr(\alpha_{i+1} \cdot \alpha_{j+1}^*)\\
& = & \delta_{ij} - \delta_{ij}\\
& = & 0
\end{IEEEeqnarray}
Hence, $\theta \alpha_i - \alpha_{i+1} \in \mathbb{Q} \{1, \alpha_1\}$ for $0 \leq i \leq n-2$.  The $i = 0$ case tells us that $\theta = a + b \alpha_1$ for some $a, b \in \mathbb{Q}$.  To ensure linear independence of $\{\alpha_1^*, \ldots, \alpha_{n-1}^* \}$, we cannot have $\theta \in \mathbb{Q}$ (at least if $n \geq 3$, as we assume), so $b \neq 0$.  Hence, we have $\mathbb{Q} \{1, \alpha_1 \} = \mathbb{Q} \{1, \theta \}$, which together with $\theta \alpha_i - \alpha_{i+1} \in \mathbb{Q} \{1, \alpha_1\}$ implies $\mathbb{Q} \{1, \alpha_1, \ldots, \alpha_{i+1} \} = \mathbb{Q} \{1, \theta, \ldots, \theta^{i+1} \}$ for all $i$.  It follows that $\theta$ is of degree $n$ over $\mathbb{Q}$ and that $\mathbb{Q}[\theta] = R \otimes \mathbb{Q}$.

Let $f(x,y) \in \mathbb{Z}[x,y]$ be a binary n-ic form such that $f(\theta,1) = 0$.  Note that $f_0 \neq 0$, since $\theta$ is of degree $n$ over $\mathbb{Q}$.  Let $R' = R_f$ with basis $\{\beta_0 = 1, \beta_1, \ldots, \beta_{n-1} \}$.  Then $R' \subseteq \mathbb{Q}[\theta] = R \otimes \mathbb{Q}$.  We will show that the change of basis from $R / \mathbb{Z}$ to $R' / \mathbb{Z}$ is by $u \in \mathbb{Q}$, and then that $R = R_{u^{-1} \cdot f}$.

Let $M \in GL_{n-1}(\mathbb{Q})$ such that the following holds:
\begin{equation}
\begin{pmatrix} 1 \\ \beta_1 \\ \ldots \\ \beta_{n-1} \end{pmatrix} = \begin{pmatrix} 1 & 0 \ldots 0 \\ * & \\ \ldots & M \\ * & \end{pmatrix} \begin{pmatrix} 1 \\ \alpha_1 \\ \ldots \\ \alpha_{n-1} \end{pmatrix}
\end{equation}

This induces the following relation between the two dual bases, where $N = (M^t)^{-1}$:
\begin{equation}
\begin{pmatrix} \beta_0^* \\ \beta_1^* \\ \ldots \\ \beta_{n-1}^* \end{pmatrix} = \begin{pmatrix} 1 & * \ldots * \\ 0 & \\ \ldots & N \\ 0 & \end{pmatrix} \begin{pmatrix} \alpha_0^* \\ \alpha_1^* \\ \ldots \\ \alpha_{n-1}^* \end{pmatrix}
\end{equation}

We know from our earlier work that there exists $s = \alpha_{n-1}^*$ such that $\alpha_i^* = \theta^{n-1-i} s$ for $i \neq 0$.  From the proof of Lemma \ref{points f0}, we similarly know that $\beta_i^* = \theta^{n-1-i} r$ for $i \neq 0$, where $r = \beta_{n-1}^*$.  Furthermore, we noted that $r,s$ are invertible.  Upon writing $t = rs^{-1}$, we get:
\begin{equation}
\begin{pmatrix} \theta^{n-2} t \\ \ldots \\ t \end{pmatrix} = N \begin{pmatrix} \theta^{n-2} \\ \ldots \\ 1 \end{pmatrix}
\end{equation}

The bottom equation tells us $t \in \mathbb{Q}\{1, \theta, \ldots, \theta^{n-2} \}$.  We prove by induction that $t \in \mathbb{Q}$.  For $0 \leq j \leq n-2$, let $S_j = \mathbb{Q}\{1, \theta, \ldots, \theta^j \}$.  Suppose that $t \in S_j$ for some $1 \leq j \leq n-2$.  We claim that $t \in S_{j-1}$:

Consider $\theta^{n-1-j} t$.  If $t \in S_j$ then $\theta^{n-1-j} t \in T_j := \mathbb{Q}\{\theta^{n-1-j}, \theta^{n-j}, \ldots, \theta^{n-1} \}$.  Looking at the $j$-th row of $N$, we also see that $\theta^{n-1-j} t \in S_{n-2}$.  By linear independence of $\{1, \theta, \ldots, \theta^{n-1} \}$, $T_j \cap S_{n-2} = \mathbb{Q} \{\theta^{n-1-j}, \theta^{n-j}, \ldots, \theta^{n-2} \}$, so $\theta^{n-1-j} t \in \mathbb{Q}\{\theta^{n-1-j}, \theta^{n-j}, \ldots, \theta^{n-2} \}$ and indeed $t \in S_{j-1}$.

Hence, $t \in \mathbb{Q}$ and it follows that $N = t I_{n-1}$, $M = t^{-1} I_{n-1}$.  We need a little extra work to confirm that $R = R_{t \cdot f}$.  There exist $b_k \in \mathbb{Q}$ such that $\alpha_k = t \beta_k + b_k$ for $1 \leq k \leq n-1$.  We need to show $b_k \in \mathbb{Z}$ for all $k$ and that $t \cdot f \in \mathbb{Z}[x,y]$.

In Nakagawa's paper he proves the following formula.  For $1 \leq i \leq j \leq n-1$ and with $\beta_n = -f_n$:
\begin{equation}
\beta_i \beta_j = \sum f_{i+j-k} \beta_k - \sum f_{i+j-k} \beta_k
\end{equation}
where the first sum is over all $k$ with $j < k \leq min(i+j,n)$ and the second sum is over all $k$ with $max(i+j-n,1) \leq k \leq i$.

We will use some special forms of this general equation.  First, note that if $1 < j < n-1$, $\beta_1 \beta_j = f_0 \beta_{j+1} - f_j \beta_1$ and $\beta_1 \beta_{n-1} = - f_0 f_n - f_{n-1} \beta_1$.  Secondly, note that $\beta_j^2$ has $\beta_j$ coefficient equal to $-f_j$, i.e. $Tr(\beta_j^2 \beta_j^*) = - f_j$.

We now calculate $\alpha_1 \alpha_j$ and $\alpha_j^2$.  For $1 < j < n-1$:
\begin{IEEEeqnarray}{rCl}
\alpha_1 \alpha_j & = & (t \beta_1 + b_1) (t \beta_j + b_j) \\
& = & t^2 (f_0 \beta_{j+1} - f_j \beta_1) + b_1 \alpha_j + b_j \alpha_1 - b_1 b_j \\
& \equiv & t f_0 \alpha_{j+1} + b_1 \alpha_j + (b_j - t f_j) \alpha_1 \text{ mod } \mathbb{Q} \\
\alpha_1 \alpha_{n-1} & = & (t \beta_1 + b_1) (t \beta_{n-1} + b_{n-1}) \\
& = & t^2 (- f_0 f_n - f_{n-1} \beta_1) + b_1 \alpha_{n-1} + b_{n-1} \alpha_1 - b_1 b_{n-1} \\
& \equiv & b_1 \alpha_{n-1} + (b_{n-1} - t f_{n-1}) \alpha_1 \text{ mod } \mathbb{Q}
\end{IEEEeqnarray}

Since $R$ is closed under multiplication, this tells us $t f_0, b_1, b_j - t f_j \in \mathbb{Z}$ for $1 < j \leq n-1$.

Now, consider $\alpha_j^2$ for $1 \leq j \leq n-1$, and recall that $\beta_j^* = t \alpha_j^*$:
\begin{IEEEeqnarray}{rCl}
\alpha_j^2 & = & (t \beta_j + b_j)^2 \\
& = & t^2 \beta_j^2 + 2 b_j \alpha_j - b_j^2 \\
\mathbb{Z} \ni Tr(\alpha_j^2 \alpha_j^*) & = & Tr(t^2 \beta_j^2 \alpha_j^*) + 2 b_j \\
& = & Tr(t \beta_j^2 \beta_j^*) + 2 b_j \\
& = & - t f_j + 2 b_j
\end{IEEEeqnarray}

Bringing this all together, we have $t f_0, b_1, b_j - t f_j \in \mathbb{Z}$ for $1 < j \leq n-1$ and $2 b_j - t f_j \in \mathbb{Z}$ for $1 \leq j \leq n-1$, so $t f_j, b_j \in \mathbb{Z}$ for all $j$.  This means that, as a based ring, $R = \mathbb{Z} \{ 1, t \beta_1, \ldots, t \beta_{n-1} \} = R_{t \cdot f}$, with $t \cdot f \in \mathbb{Z}[x,y]$ and $t \cdot f_0 \neq 0$, as we aimed to show.
\end{proof}

We give below an alternative, related characterisation of based $n$-ic rings of non-zero discriminant which come from binary $n$-ic forms.  The point of this corollary is that, not only can we ask that the $n$ points of $X_R$ are of the form $(x^{n-2},x^{n-3} y, \ldots, y^{n-2})$, we can actually ask for the dual basis to essentially be of this form, which is a priori a stronger condition:

\begin{corollary}
For $n \geq 3$, a based n-ic ring $R$ of non-zero discriminant, with basis $\{\alpha_0 = 1, \alpha_1, \ldots, \alpha_{n-1} \}$, is given by a binary n-ic form if and only if the dual basis $\{ \alpha_0^*, \alpha_1^*, \ldots, \alpha_{n-1}^* \}$ is of the form $\alpha_i^* = \lambda^{n-1-i} \mu^{i-1} r$ for $1 \leq i \leq n-1$, for some $\lambda, \mu, r \in R \otimes \mathbb{Q}$.  Furthermore, $r \in (R \otimes \mathbb{Q})^\times$.
\end{corollary}

\begin{proof}
The forward direction follows from the proof of Lemma \ref{points f0}, upon noting that the statement of this corollary is $GL_2(\mathbb{Z})$-invariant.  The reverse direction follows from Theorem \ref{n-ic theorem}.

If $r \notin (R \otimes \mathbb{Q})^\times$, there exists a homomorphism $\rho : R \otimes \mathbb{Q} \to \mathbb{C}$ with $\rho(r) = 0$.  Then $\rho(\alpha_i^*) = 0$ for all $1 \leq i \leq n-1$, contradicting Definition \ref{X_R well-defined}.
\end{proof}

\section{Quadrics vanishing on $X_{R_f}$}

A based quartic ring $R$ (with a cubic resolvent) is described by a pair of ternary quadratic forms vanishing on $X_R$.  A based quintic ring $R$ (with a sextic resolvent) is described by an element $A \in \mathbb{Z}^4 \otimes \wedge^2 \mathbb{Z}^5$, and the five sub-pfaffians of $A$ are quadrics vanishing on $X_R$.  In both these cases, these quadrics in fact form a basis for the space of quadrics vanishing on $X_R$.

Now that we know what $X_{R_f}$ looks like, we can explicitly find such quadrics in these cases and use these to lead us to the element in the relevant parameter space which describes $R_f$.

\begin{lemma} \label{rational normal quadrics}
A basis for the space of quadrics vanishing on $V_r \subseteq \mathbb{P}^r$ is given by the following quadrics:
\begin{equation}
Q_{ijkl} = t_i t_l - t_j t_k
\end{equation}
with $i < j \leq k < l, i+l = j+k$, $i=0$ if $s \leq r$ and $l = r$ if $s > r$.
\end{lemma}

\begin{proof}
$V_r$ is the collection of elements of the form $q_r(x,y) = ( x^r, x^{r-1} y, \ldots, y^r )$.  Let $Q$ be a quadric vanishing on all such elements.  Evaluating $Q$ at this generic element, we obtain a degree $2r$ polynomial in $x,y$.  We can partition $Q$ into sums of terms $t_i t_j$ with $i+j$ fixed, which correspond to the monomial $x^{2r-i-j} y^{i+j}$.  For a fixed value $s = i+j$, let $Q(s)$ be the sum of terms in $Q$ with $i+j = s$.  We need the sum of the coefficients in each $Q(s)$ to equal 0.

If $s \in \{ 0,1,2r-1,2r \}$, there is only one monomial $t_i t_j$ in $Q(s)$ (respectively, $t_0^2, t_0 t_1, t_{r-1} t_r, t_r^2$), so the coefficients of these terms are all 0.  If $s \leq r$, the vanishing property of $Q(s)$ means it has to be in the span of $Q_{0ijs}$ with $0 < i \leq j < s$ and $i+j = s$, and these quadrics are seen to be linearly independent.  If $s > r$, $Q(s)$ is in the span of $Q_{(s-r)ijr}$ with $s-r < i \leq j < r$ and $i+j = s$.  Gathering these various quadrics, we retain linear independence and obtain the given basis.
\end{proof}

\begin{lemma} \label{R_f quadrics}
A basis for the space of quadrics vanishing on $X_{R_f} \subseteq \mathbb{P}^{n-2}$ is given by the quadrics of Lemma \ref{rational normal quadrics} together with the following quadrics:
\begin{equation}
Q_i = f_0 t_0 t_i + f_1 t_0 t_{i+1} + \ldots + f_{n-2-i} t_0 t_{n-2} + f_{n-1-i} t_1 t_{n-2} + \ldots + f_n t_{i+2} t_{n-2}
\end{equation}
for $0 \leq i \leq n-4$.
\end{lemma}

\begin{proof}
Recall from Theorem \ref{n-ic theorem} that $X_{R_f} = \{ q_{n-2}(\lambda , \mu) : f( \lambda, \mu ) = 0 \}$.  In particular, it is contained in $V_{n-2}$ so the quadrics in Lemma \ref{rational normal quadrics} certainly vanish on $X_{R_f}$.

To see that all other quadrics vanishing on $X_{R_f}$ are captured by the new quadrics $Q_i$, we first check that the $Q_i$ do indeed vanish on $X_{R_f}$:
\begin{IEEEeqnarray}{rCl}
Q_i( q_{n-2}(x,y) ) & = & f_0 x^{2n-4-i} y^i + \ldots + f_n x^{n-4-i} y^{n+i} \\
& = & x^{n-4-i} y^i \cdot f(x,y)
\end{IEEEeqnarray}

Hence, the $Q_i$ vanish on $X_{R_f}$.  From the set $\{ Q_i( q_{n-2}(x,y) ) : 0 \leq i \leq n-4 \}$ and the basis of Lemma \ref{rational normal quadrics} both being linearly independent, we see that the proposed basis $\{ Q_{ijkl} \} \cup \{ Q_i \}$ is linearly independent.

It remains to check that any quadric vanishing on $X_{R_f}$ is in their span.  Let $Q$ be such a quadric.  Then $Q( q_{n-2}(x,y) )$ is a polynomial in $x,y$ vanishing at the roots of $f(x,y)$.  Since $f$ has non-zero discriminant, it is square free and so $Q( q_{n-2}(x,y) )$ is a multiple of $f(x,y)$.  Thus, $Q( q_{n-2}(x,y) ) = \sum_i a_i Q_i ( q_{n-2}(x,y) )$ for some $a_i$, and $Q - \sum_i a_i Q_i$ is a quadric vanishing on $V_{n-2}$, so it is in the span of $\{ Q_{ijkl} \}$.
\end{proof}

Applying this result in the case of binary quartic forms, we are led to a pair of forms which are essentially those derived by Wood:

\begin{corollary}
Let $f = f_0 x^4 + f_1 x^3 y + \ldots + f_4 y^4$ of non-zero discriminant, and let $R_f$ be given by a pair of ternary quadratic forms $(A,B)$.  Then $A,B$ lie in the span of the forms $\{ A', B' \}$ below:
\begin{IEEEeqnarray}{rCl}
A' & = & t_0 t_2 - t_1^2 \\
B' & = & f_0 t_0^2 + f_1 t_0 t_1 + f_2 t_0 t_2 + f_3 t_1 t_2 + f_4 t_2^2
\end{IEEEeqnarray}

These can be written as the following two matrices:
\begin{equation}
\begin{pmatrix}
0 & 0 & \frac{1}{2} \\
0 & -1 & 0 \\
\frac{1}{2} & 0 & 0
\end{pmatrix},
\begin{pmatrix}
f_0 & \frac{f_1}{2} & \frac{f_2}{2} \\
\frac{f_1}{2} & 0 & \frac{f_3}{2} \\
\frac{f_2}{2} & \frac{f_3}{2} & f_4
\end{pmatrix}
\end{equation}
\end{corollary}

\begin{proof}
The forms $A,B$ span the quadrics vanishing on $X_{R_f}$, and so this follows from the previous result.
\end{proof}

Note: It is not immediate that this pair of forms indeed give rise to the ring $R_f$.  A priori, they could give rise to some other ring in the same etale algebra.  However, the result is still useful in that it leads us in the right direction.

Similarly, we can say the following in the quintic case:

\begin{corollary} \label{sub-pfaffians}
Let $f = f_0 x^5 + f_1 x^4 y + \ldots + f_5 y^5$ of non-zero discriminant, and let $R_f$ be given by $A \in \mathbb{Z}^4 \otimes \wedge^2 \mathbb{Z}^5$.  With the coordinates of $\mathbb{P}^3$ being $t_1,t_2,t_3,t_4$, the sub-pfaffians of $A$ lie in the span of the forms below:
\begin{IEEEeqnarray}{rCl}
P_1 & = & t_1 t_3 - t_2^2\\
P_2 & = & t_1 t_4 - t_2 t_3\\
P_3 & = & t_2 t_4 - t_3^2\\
P_4 & = &  f_0 t_1 t_2 +  f_1 t_2^2 +  f_2 t_2 t_3 +  f_3 t_3^2 +  f_4 t_3 t_4 +  f_5 t_4^2\\
P_5 & = &  f_0 t_1^2 +  f_1 t_1 t_2 +  f_2 t_2^2 +  f_3 t_2 t_3 +  f_4 t_3^2 +  f_5 t_3 t_4
\end{IEEEeqnarray}
\end{corollary}

\begin{proof}
The sub-pfaffians of $A$ span the space of quadrics vanishing on $X_{R(A)} = X_{R_f}$.  This result then follows from Lemma \ref{R_f quadrics} with some slight changes to the precise basis chosen.
\end{proof}

This result will be key in the next chapter, where we closely analyse the quintic rings given by binary quintic forms.

\chapter{Maps from $(Sym^5 \mathbb{Z}^2)^* \to \mathbb{Z}^4 \otimes \wedge^2 \mathbb{Z}^5$}

The maps below mimic closely the maps from binary quartics to $\mathbb{Z}^2 \otimes Sym^2 \mathbb{Z}^3$ defined by Wood.

Denote a generic element $f \in (Sym^5 \mathbb{Z}^2)^*$ by $f = f_0 x^5 + f_1 x^4 y + f_2 x^3 y^2 + f_3 x^2 y^3 + f_4 x y^4 + f_5 y^5$.

\begin{theorem} \label{Phi}
The following map $\Phi: (Sym^5 \mathbb{Z}^2)^* \to \mathbb{Z}^4 \otimes \wedge^2 \mathbb{Z}^5$ respects the two constructions of based quintic rings:\\

$\Phi(f) = \begin{pmatrix}
0 & t_3 & - t_2 & t_1 & 0\\
- t_3 & 0 & -  f_0 t_1 -  f_1 t_2 & -  f_2 t_2 -  f_3 t_3 & - t_4\\
t_2 &  f_0 t_1 +  f_1 t_2 & 0 & -  f_4 t_3 -  f_5 t_4 & t_3\\
- t_1 &  f_2 t_2 +  f_3 t_3 &  f_4 t_3 +  f_5 t_4 & 0 & - t_2\\
0 & t_4 & - t_3 & t_2 & 0\\
\end{pmatrix}$\\

\end{theorem}
\begin{proof}
Computation of the $SL_5$-invariants leads to the multiplicative structure of $R(\Phi(f))$, which is seen to match that of the ring $R_f$ defined by Birch and Merriman.
\end{proof}

However, the derivation of this map was not entirely guesswork.  It relies on noticing that the sub-pfaffians can be taken to be quite simple quadrics as explained by our earlier result, Corollary \ref{sub-pfaffians}, which we restate below.

\begin{corollary}
Let $f = f_0 x^5 + f_1 x^4 y + \ldots + f_5 y^5$ of non-zero discriminant, and let $R_f$ be given by $A \in \mathbb{Z}^4 \otimes \wedge^2 \mathbb{Z}^5$.  With the coordinates of $\mathbb{P}^3$ being $t_1,t_2,t_3,t_4$, the sub-pfaffians of $A$ lie in the span of the forms below:
\begin{IEEEeqnarray}{rCl}
P_1 & = & t_1 t_3 - t_2^2\\
P_2 & = & t_1 t_4 - t_2 t_3\\
P_3 & = & t_2 t_4 - t_3^2\\
P_4 & = &  f_0 t_1 t_2 +  f_1 t_2^2 +  f_2 t_2 t_3 +  f_3 t_3^2 +  f_4 t_3 t_4 +  f_5 t_4^2\\
P_5 & = &  f_0 t_1^2 +  f_1 t_1 t_2 +  f_2 t_2^2 +  f_3 t_2 t_3 +  f_4 t_3^2 +  f_5 t_3 t_4
\end{IEEEeqnarray}
\end{corollary}

This knowledge of the sub-pfaffians simplifies and guides the brute-force search for $\Phi$.  The signed sub-pfaffians of $\Phi(f)$ turn out to be $[ P_4, P_1, P_2, P_3, - P_5 ]$.

We begin investigating the map $\Phi$ and list key properties below.

\begin{corollary} \label{disc}
The map $\Phi$ is discriminant-preserving.
\end{corollary}
\begin{proof}
For $A \in \mathbb{Z}^4 \otimes \wedge^2 \mathbb{Z}^5$, the discriminant of $R(A)$ was shown by Bhargava to be $Disc(A)$.  Similarly, Birch and Merriman showed that the discriminant of $R_f$ is $\Delta(f)$.  The result then follows from Theorem \ref{Phi}.
\end{proof}

\begin{theorem} \label{GL2}
If $\gamma \in GL_2(\mathbb{Z})$, then $\Phi(\gamma \cdot f)$ and $\Phi(f)$ are $GL_4(\mathbb{Z}) \times GL_5(\mathbb{Z})$-equivalent.

Explicitly, define $\sigma :  GL_2(\mathbb{Z}) \to GL_4(\mathbb{Z}) \times GL_5(\mathbb{Z})$ as follows:
\begin{IEEEeqnarray}{rCl}
\gamma & = & \begin{pmatrix} a & b \\ c & d \end{pmatrix}\\
\sigma(\gamma) & = & (\psi(\gamma),\rho(\gamma))\\
\psi(\gamma) & = & (ad-bc) \begin{pmatrix}
a^3 & a^2 b & a b^2 & b^3 \\
3 a^2 c & 2 a b c + a^2 d & 2 a b d + b^2 c & 3 b^2 d\\
3 a c^2 & 2 a c d + b c^2 & 2 b c d + a d^2 & 3 b d^2\\
c^3 & c^2 d & c d^2 & d^3
\end{pmatrix}\\
\rho(\gamma) & = & \begin{pmatrix}
d & 0 & 0 & 0 & - c \\
0 & a^2 & 2ab & b^2 & 0 \\
0 & ac & bc+ad & bd & 0 \\
0 & c^2 & 2cd & d^2 & 0 \\
- b & 0 & 0 & 0 & a
\end{pmatrix}
\end{IEEEeqnarray}

Recall that $H \leqslant SL_5(\mathbb{Z})$ is the group of all matrices of the form:

\begin{equation}
\begin{pmatrix}
1 & 0 & 0 & 0 & 0 \\
* & 1 & 0 & 0 & * \\
* & 0 & 1 & 0 & * \\
* & 0 & 0 & 1 & * \\
0 & 0 & 0 & 0 & 1
\end{pmatrix}
\end{equation}

Then, there exists $h \in H$ such that $\Phi( \gamma \cdot f) = (1,h) \cdot \sigma ( \gamma ) \cdot \Phi(f)$.

\end{theorem}

\begin{proof}
This can be checked by explicit computation, but the derivation of the result in the first place is illuminating.

Let $g = \gamma \cdot f$.  Nakagawa proved that $R_f = R_g$, with their bases related (mod $\mathbb{Z}$) by $\psi(\gamma)$.  Hence, if $\Phi(g) = ( \alpha , \beta ) \cdot \Phi(f)$ for some $( \alpha , \beta ) \in \Gamma$, then $\alpha = \psi ( \gamma )$.

To find a candidate for $\beta$, we consider the action on sub-pfaffians.  Denote the five $4 \times 4$ signed sub-pfaffians of $\Phi(f)$ by $P_{1,f}, \ldots, P_{5,f}$, and do the same for $g$.  If $\Phi(g) = ( \alpha, \beta ) \cdot \Phi(f)$ then Higher Composition Laws IV tells us that:

\begin{equation}
\begin{pmatrix}
P_{1,g} \\ P_{2,g} \\ P_{3,g} \\ P_{4,g} \\ P_{5,g}
\end{pmatrix}
=
(\det \beta ) (\beta^{-1})^t
\begin{pmatrix}
\alpha P_{1,f} \alpha^t \\\alpha  P_{2,f} \alpha^t \\ \alpha P_{3,f} \alpha^t \\ \alpha P_{4,f} \alpha^t \\ \alpha P_{5,f} \alpha^t
\end{pmatrix}
\end{equation}

Using $\alpha = \psi ( \gamma )$, we compute this all and see that $\beta$ would have to be of the form $h \cdot \rho( \gamma )$ for a unique $h$.

Then, we check with this $\beta$ that indeed $\Phi(g) = (\alpha,\beta) \cdot \Phi(f)$.
\end{proof}

There is a relative $\Phi'$ of $\Phi$, which is $GL_2(\mathbb{Z})$-equivariant on the nose (i.e. not up to some $h \in H$), but is only defined for integer-matrix quintic forms.  They are related in the sense that $\Phi(f)$ and $\Phi'(f)$ are $GL_5(\mathbb{Z})$-translates:
\begin{prop}
There exists $\Phi' : Sym^5 \mathbb{Z}^2 \to \mathbb{Z}^4 \otimes \wedge^2 \mathbb{Z}^5$ and $\sigma: GL_2(\mathbb{Z}) \to GL_4(\mathbb{Z}) \times GL_5(\mathbb{Z})$ as above, such that $\Phi'(\gamma \cdot f) = \sigma(\gamma) \cdot \Phi'(f)$.  We also have that $R(\Phi'(f)) = R_f$ as based rings.

In full detail, $\Phi(f) = t_1 A_1 + t_2 A_2 + t_3 A_3 + t_4 A_4$ for:

\begin{IEEEeqnarray}{rCl}
A_1 & = & \begin{pmatrix}
0 & 0 & 0 & 1 & 0 \\
& 0 & -  f_0 & - \frac{2 f_1}{5} & 0 \\
& & 0 & - \frac{ f_2}{10} & 0 \\
& & & 0 & 0 \\
& & & & 0
\end{pmatrix} \\
A_2 & = & \begin{pmatrix}
0 & 0 & - 1 & 0 & 0 \\
& 0 & - \frac{3  f_1}{5} & - \frac{3  f_2}{5} & 0 \\
& & 0 & - \frac{3  f_3}{10} & 0 \\
& & & 0 & -1 \\
& & & & 0
\end{pmatrix} \\
A_3 & = & \begin{pmatrix}
0 & 1 & 0 & 0 & 0 \\
& 0 & - \frac{3  f_2}{10} & - \frac{3  f_3}{5} & 0 \\
& & 0 & - \frac{3  f_4}{5} & 1 \\
& & & 0 & 0 \\
& & & & 0
\end{pmatrix} \\
A_4 & = & \begin{pmatrix}
0 & 0 & 0 & 0 & 0 \\
& 0 & - \frac{ f_3}{10} & - \frac{2  f_4}{5} & -1 \\
& & 0 & -  f_5 & 0 \\
& & & 0 & 0 \\
& & & & 0
\end{pmatrix}
\end{IEEEeqnarray}

\end{prop}

\begin{proof}
We aim to find a $GL_5(\mathbb{Z})$-translate of $\Phi(f)$, which respects the $GL_2(\mathbb{Z})$ action on the nose and does not need slight adjustment by $h \in H$.  Thus, we look for a polynomial map $P(f) \in H$ such that $\Phi'(f) = P(f) \cdot \Phi(f)$ is $GL_2(\mathbb{Z})$-equivariant, i.e. we want $\Phi'( \gamma \cdot f ) = \sigma( \gamma ) \cdot \Phi'( f )$.

Our work from Theorem \ref{GL2} in fact tells us that $\Phi( \gamma \cdot f) = (1,h(\gamma,f)) \cdot \sigma(\gamma) \cdot \Phi(f)$, for $h(\gamma,f)$ polynomial in $\gamma$ and $f$.  We calculate:
\begin{IEEEeqnarray}{rCl}
\Phi'( \gamma \cdot f ) & = & P( \gamma \cdot f ) \cdot \Phi( \gamma \cdot f ) \\
& = & P( \gamma \cdot f ) \cdot (1,h(\gamma,f)) \cdot \sigma(\gamma) \cdot \Phi(f) \\
& = & P( \gamma \cdot f ) \cdot (1,h(\gamma,f)) \cdot \sigma(\gamma) \cdot P(f)^{-1} \Phi'(f) \\
& = & ( \psi(\gamma), P( \gamma \cdot f ) h(\gamma,f) \rho(\gamma) P(f)^{-1} ) \cdot \Phi'(f)
\end{IEEEeqnarray}

Thus, we see that we need $P$ such that $P( \gamma \cdot f ) h(\gamma,f) \rho(\gamma) P(f)^{-1} = \rho(\gamma)$.  From explicit calculation, the following $P$ arises and indeed does the job, leading to $\Phi'(f)$:
\begin{equation}
P(f) =
\begin{pmatrix}
1 & 0 & 0 & 0 & 0\\ - \frac{2}{5}  f_1 & 1 & 0 & 0 & \frac{2}{5}  f_2 \\ - \frac{1}{10} f_2 & 0 & 1 & 0 & \frac{1}{10} f_3 \\ - \frac{2}{5} f_3 & 0 & 0 & 1 & \frac{2}{5} f_4 \\ 0 & 0 & 0 & 0 & 1
\end{pmatrix}
\end{equation}
\end{proof}

This map is only defined for integer-matrix forms, but there is a third, related map which respects the action of $GL_2(\mathbb{Z})$ and is defined for all binary quintic forms.  Inspired by the analogous map in the case of binary quartics, we can map to a quotient of $\mathbb{Z}^4 \otimes \wedge^2 \mathbb{Z}^5$:
\begin{corollary}
Define $\bar{\Phi} : Sym^5 \mathbb{Z}^2 \to H \backslash (\mathbb{Z}^4 \otimes \wedge^2 \mathbb{Z}^5)$ by $\bar{\Phi}(f) = H \Phi(f)$.  Then $\bar{\Phi}(\gamma \cdot f) = \sigma(\gamma) \cdot \bar{\Phi}(f)$.
\end{corollary}
\begin{proof}
Follows from Theorem \ref{GL2}.  [Note that this claim is well-defined: $\sigma(\gamma) \cdot \bar{\Phi}(f)$ is a left $H$-coset because $\sigma(\gamma) H \sigma(\gamma)^{-1} = H$.]
\end{proof}

\begin{corollary}
The maps $\Phi'$ and $\bar{\Phi}$ are both discriminant-preserving.
\end{corollary}
\begin{proof}
The based quintic ring in question is unchanged, so the same proof as that of Corollary \ref{disc} applies.
\end{proof}








\chapter{The sextic resolvent ring $S_f$} \label{sextic section}

Binary quartic forms have the special property that they describe quartic rings whose cubic resolvent is monogenic.  Given a binary quintic form, with associated quintic ring and sextic resolvent, what special properties does the resolvent ring have?

For binary quartic forms, the basis of the cubic resolvent ring depends only on the $GL_2(\mathbb{Z})$-orbit of the binary quartic in question.  This is because the relevant map $GL_2(\mathbb{Z}) \to GL_2(\mathbb{Z}) \times GL_3(\mathbb{Z})$ has trivial $GL_2(\mathbb{Z})$ component.  This means that the structure of the cubic resolvent can be understood in terms of the $GL_2$ invariants of the binary quartic form.

For binary quintic forms, since the map $\sigma$ lands non-trivially in $GL_5(\mathbb{Z})$, this is not the case; just as the structure coefficients of the based ring $R_f$ depend on $f$ and not just its equivalence class, so the same will be true for the sextic resolvent $S_f = S(\Phi(f))$.

Yet, the block matrix form of $\rho(\gamma)$ is intriguing, and suggests there may be special structure in the resolvent ring; indeed, this has to do with the relation between $\{ \beta_1^*, \beta_5^* \}$ and $\{ \beta_2, \beta_3, \beta_4 \}$.

\section{The structure of the sextic resolvent ring}

Recall the definition of dual-digenicity:
\begin{definition} \label{dual digenic2}
Let $(R,S)$ be a quintic ring / sextic resolvent pair, with both rings of non-zero discriminant.  Consider $S$ as a based ring, with basis denoted by $\{1,\beta_1,\ldots,\beta_5\}$.  Denote the dual basis with respect to the trace pairing by $\{\beta_0^*,\beta_1^*,\ldots,\beta_5^*\}$, i.e. $\beta_i^* \in S \otimes \mathbb{Q}$ such that $Tr(\beta_i^* \beta_j) = \delta_{ij}$.

Then, we say that the based ring $S$ is \emph{dual-digenic} if it has the following property:
\begin{equation}
(\beta_2,\beta_3,\beta_4) \equiv 8 \, Disc(R) \cdot ( (\beta_1^*)^2, 2 \beta_1^* \beta_5^*, (\beta_5^*)^2 ) \, \text{mod} \, \mathbb{Q}
\end{equation}

This can also be rephrased in terms of $Disc(S)$ using $Disc(S) = (16 \cdot Disc(R))^3$.

We also define a (non-based) ring $S$ to be dual-digenic if it has a basis in which it is dual-digenic.
\end{definition}

A few notes on this definition:
\begin{itemize}
\item That $S$ is based means that the $\beta_i$ are only determined mod $\mathbb{Z}$.  However, the $\beta_i^*$ will then be completely determined.  Thus, the notion of dual-digenicity is well-defined.
\item Since the basis of $S$ is determined only mod $\mathbb{Z}$, it wouldn't make sense for this definition to be stated on the nose with an equality.
\item But, then why is it stated mod $\mathbb{Q}$ and not mod $\mathbb{Z}$?  We will see that, for $S = S_f$, the three expressions $8 \Delta(f) \cdot ((\beta_1^*)^2, 2 \beta_1^* \beta_5^*, (\beta_5^*)^2)$ may not lie in $S$, but some $\mathbb{Q}$-translate of each of them does.  These will in turn be congruent to $\beta_2, \beta_3, \beta_4$ mod $\mathbb{Z}$.
\end{itemize}


\begin{theorem} \label{sextic structure2}
Let $f$ be a binary quintic form of non-zero discriminant $\Delta(f)$.  Then, the based ring $S_f$ is dual-digenic.
\end{theorem}
\begin{proof}
Proof is by computation of the $SL_4$-invariants of $\Phi(f)$, from which the multiplicative structure of $S_f$ can be computed as explained in Higher Composition Laws IV.  [See Appendix for multiplication table.]
\end{proof}

[Note:   If we had a different way to prove the Segre cubic = trace cubed property of Theorem \ref{trace cubed}, that would result in an alternative proof of the above result.  Currently, the proof of Theorem \ref{trace cubed} also relies on the multiplication table of $S_f$, though it is true for all $S(A)$.]

In a based sextic ring $S_f$, we shall use the shorthand $x = \beta_1^*, y = \beta_5^*$.  Dual-digenicity of $S_f$ tells us the following about $x,y$:
\begin{lemma} $Tr(x^3) = Tr(x^2 y) = Tr(x y^2) = Tr(y^3) = 0$
\end{lemma}
\begin{proof}
Since $Tr(\beta_i \beta_j^*) = 0$ for $i \neq j$ and $Tr(\beta_j^*) = 0$ for $j \neq 0$, choosing $i \in \{2,3,4\}$, $j \in \{1,5\}$ and applying Theorem \ref{sextic structure2} does the job.
\end{proof}

\begin{lemma} \label{5th powers}
$(8 \Delta(f))^2 \cdot Tr(x^i y^j) \in \mathbb{Z}$ for $i,j \geq 0, i + j = 5$.
\end{lemma}
\begin{proof}
For $i,j \in \{2,3,4\}$, $k \in \{1,5\}$, $\mathbb{Z} \ni d_{ij}^k = Tr(\beta_i \beta_j \beta_k^*)$ and this expression is invariant under translating $\beta_i, \beta_j$ by elements of $\mathbb{Q}$.  Hence:
\begin{IEEEeqnarray}{rClCl}
d_{22}^1 & = & (8 \Delta(f))^2 & \cdot & Tr(x^5)\\
d_{22}^5 & = & (8 \Delta(f))^2 & \cdot & Tr(x^4 y)\\
d_{23}^1 & = & 2 (8 \Delta(f))^2 & \cdot & Tr(x^4 y)\\
d_{23}^5 & = & 2 (8 \Delta(f))^2 & \cdot & Tr(x^3 y^2)\\
d_{24}^1 & = & (8 \Delta(f))^2 & \cdot & Tr(x^3 y^2)\\
d_{24}^5 & = & (8 \Delta(f))^2 & \cdot & Tr(x^2 y^3)\\
d_{33}^1 & = & 4 (8 \Delta(f))^2 & \cdot & Tr(x^3 y^2)\\
d_{33}^5 & = & 4 (8 \Delta(f))^2 & \cdot & Tr(x^2 y^3)\\
d_{34}^1 & = & 2 (8 \Delta(f))^2 & \cdot & Tr(x^2 y^3)\\
d_{34}^5 & = & 2 (8 \Delta(f))^2 & \cdot & Tr(x y^4)\\
d_{44}^1 & = & (8 \Delta(f))^2 & \cdot & Tr(x y^4)\\
d_{44}^5 & = & (8 \Delta(f))^2 & \cdot & Tr(y^5)
\end{IEEEeqnarray}
\end{proof}

\begin{lemma} \label{recovery}
$Tr ( (u \beta_5^* - v \beta_1^*)^5) = - 10 f(u,v)$
\end{lemma}
\begin{proof}
Finding the structure coefficients $d_{ij}^k$ in the Appendix, we see that for $i,j \in \{2,3,4\}, k \in \{1,5\}$, they are all integer multiples of some $f_i$:
\begin{IEEEeqnarray}{rClCrCl}
d_{22}^1 & = & 10 f_5 & \, || \, & d_{22}^5 & = & - 2 f_4\\
d_{23}^1 & = & - 4 f_4 & \, || \, & d_{23}^5 & = & 2 f_3\\
d_{24}^1 & = & f_3 & \, || \, & d_{24}^5 & = & - f_2\\
d_{33}^1 & = & 4 f_3 & \, || \, & d_{33}^5 & = & - 4 f_2\\
d_{34}^1 & = & - 2 f_2 & \, || \, & d_{34}^5 & = & 4 f_1\\
d_{44}^1 & = & 2 f_1 & \, || \, & d_{44}^5 & = & - 10 f_0
\end{IEEEeqnarray}
Lemma \ref{5th powers} then completes the proof.
\end{proof}

\begin{corollary}
A based sextic ring $S$ can arise from at most one binary quintic form as $S_f = S(\Phi(f))$.
\end{corollary}
\begin{proof}
If $S = S_f$ for some $f$, then $f$ is encoded in $\beta_1^*, \beta_5^*$ as in Lemma \ref{recovery}.
\end{proof}

\section{An associated Segre cubic threefold}

There is a variety called a Segre cubic 3-fold attached to an element $A \in \mathbb{C}^4 \otimes \wedge^2 \mathbb{C}^5$.  This is work of Seok Hyeong (Sean) Lee [All original work?  If not, which parts are due to Seok Hyeong?].  The construction is as follows:

Given a quadruple $A = (A_1, A_2, A_3, A_4)$ of skew-symmetric $5 \times 5$ matrices, for each point $x = (x_1:x_2:x_3:x_4)$ in $\mathbb{P}^3$, the $5 \times 5$ skew-symmetric matrix $x_1 A_1 + x_2 A_2 + x_3 A_3 + x_4 A_4$ has even rank.  We denote this matrix by $A(x)$.  For generic $x \in \mathbb{P}^3$, it has rank 4, but for bad $x$ - which turn out to be the five points of $X_{R(A)}$ - its rank will be 2.  Consider the following subvariety of $\mathbb{P}^3 \times \mathbb{P}^4$: $V = \{ (x,y) : y \in ker(A(x)) \}$.  When we consider the map $V \to \mathbb{P}^3$, away from the five bad points identified, there is a unique point above each point $x \in \mathbb{P}^3$, since $dim(ker(A(x))) = 1$ generically.  At the five bad points, by virtue of $dim(ker(A(x))) = 3$, there is a plane above each such point.  So, $V$ is $\mathbb{P}^3$ blown up at five points.  If we now consider the image of $V$ in $\mathbb{P}^4$, it turns out that this is a Segre cubic 3-fold, which is a dimension 3 cubic variety with ten singularities, which are all nodal.

\begin{theorem} [S. H. Lee?] \label{segre equation}
Let $A \in \mathbb{C}^4 \otimes \wedge^2 \mathbb{C}^5$, $y,z \in \mathbb{C}^5$ with $y = (y_1, y_2, y_3, y_4, y_5)^t$ and $z = (z_1, z_2, z_3, z_4, z_5)^t$.  The determinant $A_1 y \wedge A_2 y \wedge A_3 y \wedge A_4 y \wedge z$ factors as $\langle y,z \rangle \cdot F_A(y)$, where $\langle y,z \rangle = \sum y_i z_i$ is the usual bilinear dot product and $F_A(y)$ is a cubic form in $y$.  Then, $F_A$ cuts out the Segre cubic threefold associated to $A$.
\end{theorem}

\begin{lemma} \label{segre action}
Let $(g,h) \in GL_4(\mathbb{C}) \times GL_5(\mathbb{C})$, and let $A' = (g,h) \cdot A$.  Then:
\begin{equation}
F_{A'}(y) = \det(g) \det(h) \, F_A(h^t y)
\end{equation}
\end{lemma}

\begin{proof}
\begin{IEEEeqnarray}{rCl}
\langle y,z \rangle \cdot F_{A'}(y) & = & A'_1 y \wedge A'_2 y \wedge A'_3 y \wedge A'_4 y \wedge z \\
& = & \det(g) \cdot \nonumber \\
& & \quad (hAh^t)_1 y \wedge (hAh^t)_2 y \wedge (hAh^t)_3 y \wedge (hAh^t)_4 y \wedge z \\
& = & \det(g) \det(h) \cdot \nonumber \\
& & \quad (Ah^t)_1 y \wedge (Ah^t)_2 y \wedge (Ah^t)_3 y \wedge (Ah^t)_4 y \wedge h^{-1} z \\
& = & \det(g) \det(h) \langle h^t y, h^{-1} z \rangle \cdot  F_A(h^t y) \\
& = & \det(g) \det(h) \langle y, z \rangle \cdot F_A(h^t y)
\end{IEEEeqnarray}
\end{proof}

Before we state how the form $F_A$ is related to the arithmetic of the sextic resolvent ring, we take a moment to note that the formulae of Higher Composition Laws IV define a commutative, associative sextic $\mathbb{C}$-algebra $S(A)$ when we extend them to $A \in \mathbb{C}^4 \otimes \wedge^2 \mathbb{C}^5$.  Furthermore, $SL_4(\mathbb{C}) \times SL_5(\mathbb{C})$ acts on the basis of $S(A) / \mathbb{C}$ in the same way that $SL_4(\mathbb{Z}) \times SL_5(\mathbb{Z})$ acts on the basis of $S(A) / \mathbb{Z}$ for integral $A$.

Now we can state the meaning of $F_A$ in the sextic algebra $S(A)$:

\begin{theorem} \label{trace cubed}
Let $A, F_A, y$ be as in Theorem \ref{segre equation}, but with $Disc(A) \neq 0$.  The sextic resolvent algebra $S = S(A)$ comes equipped with a basis of $S / \mathbb{C}$, and a corresponding dual basis of $\tilde{S}$ which we denote by $\{ \beta_1^*, \ldots, \beta_5^* \}$.  Then:
\begin{equation}
8 \, Disc(A) \cdot Tr( (y_1 \beta_1^* + \ldots + y_5 \beta_5^*)^3 ) = - 3 \, F_A(y)
\end{equation}
\end{theorem}

\begin{proof}
The $G = GL_4(\mathbb{C}) \times GL_5(\mathbb{C})$ representation $\mathbb{C}^4 \otimes \wedge^2 \mathbb{C}^5$ is a prehomogeneous vector space, meaning that is has a dense open orbit.  This orbit is comprised of the elements of non-zero discriminant.  So, if we can prove that the equation is $G$-invariant and that it holds for some $A$ of non-zero discriminant, then it will hold for all $A$ of non-zero discriminant, as desired.

Taking $A = \Phi(f)$ for any binary quintic of non-zero discriminant, our knowledge of the structure coefficients of $S(A)$ enables us to prove explicitly that this formula holds.

Now to prove $G$-invariance.  Let $(g,h) \in G$ and $A' = (g,h) \cdot A$.  We know from Lemma \ref{segre action} that $F_{A'}(y) = \det(g) \det(h) \, F_A(h^t y)$.  We will prove that the left hand side of the equation transforms in the same way, by looking at the actions of $SL_4(\mathbb{C}) \times SL_5(\mathbb{C})$ and the scalar matrices $(\lambda I_4,\mu I_5)$ separately.

Let $(g,h) \in SL_4(\mathbb{C}) \times SL_5(\mathbb{C})$.  From our note above on the action of $SL_4(\mathbb{C}) \times SL_5(\mathbb{C})$ on the basis of $S(A)/\mathbb{C}$, we know that the action of $(g,h)$ on the basis $\{\beta_1^*,\ldots,\beta_5^*\}$ of $\tilde{S(A)}$ is just by $h$.  Transferring this change of basis to a transformation on $y$ amounts to replacing $y$ by $h^t y$.  The polynomial $Disc(A)$ is invariant under $SL_4(\mathbb{C}) \times SL_5(\mathbb{C})$.  Thus, both sides of the equation transform in the same way: by applying $h^t$ to $y$.

Consider the action of $(\lambda I_4,\mu I_5)$ on the equation.  The structure coefficients of the sextic resolvent algebra are degree 12 polynomials in the entries of $A$.  This means that $\lambda I_4$ acts on them by $\lambda^{12}$ and $\mu I_5$ acts by $\mu^{24}$.  Hence, $(\lambda I_4, \mu I_5)$ act on the basis of the sextic resolvent algebra by $\lambda^{12} \mu^{24}$, and so the action on the dual basis is by $\lambda^{-12} \mu^{-24}$.  The polynomial $Disc(A)$ is degree 40 and so $(\lambda,\mu)$ acts by $\lambda^{40} \mu^{80}$.  Thus $(\lambda,\mu)$ acts on the left hand side of the equation by $\lambda^4 \mu^8$.  The action on the right hand side results in $\det(\lambda I_4) \det (\mu I_5) F_A(\mu y) = \lambda^4 \mu^8 F_A(y)$ because $F_A$ is cubic, so we have equality.
\end{proof}

Let's compute the equation of the Segre cubic associated to $\Phi(f)$, which we denote by $F_f$:
\begin{IEEEeqnarray}{rCl}
F_f(y) & = & - f_0 f_2 y_2^3 - f_0 f_3 y_2^2 y_3 - f_0 f_4 y_2 y_3^2 + f_0 f_5 y_2 y_3 y_4 \nonumber \\
&& - f_0 f_5 y_3^3 + f_0 y_2^2 y_5 - f_1 f_3 y_2^2 y_4 - f_1 f_4 y_2 y_3 y_4 \nonumber \\
&& - f_1 f_5 y_3^2 y_4 - f_1 y_1 y_2^2 - f_2 f_4 y_2 y_4^2 - f_2 f_5 y_3 y_4^2 \nonumber \\
&& + f_2 y_2 y_4 y_5 - f_3 f_5 y_4^3 - f_3 y_1 y_2 y_4 + f_4 y_4^2 y_5 \nonumber \\
&& - f_5 y_1 y_4^2 - y_1^2 y_2 - y_1 y_3 y_5 - y_4 y_5^2
\end{IEEEeqnarray}

\begin{corollary} \label{segre basis property}
Let $A \in \mathbb{Z}^4 \otimes \wedge^2 \mathbb{Z}^5$.  Then $S(A)$ is dual-digenic if and only if $F_A$ takes the following form:
\begin{equation}
F_A(y) = - y_1^2 y_2 - y_1 y_3 y_5 - y_4 y_5^2 + \text{(lower order terms in $y_1, y_5$)}
\end{equation}
\end{corollary}
\begin{proof}
The product $\beta_i^* \beta_j^* \in S \otimes \mathbb{Q}$ has an expansion in terms of the basis $\{1,\beta_1, \ldots,\beta_5\}$ of $S \otimes \mathbb{Q}$.  The $\beta_k$ coefficient is given by $Tr( \beta_i^* \beta_j^* \beta_k^*)$.  From Theorem \ref{trace cubed}, we know that these expressions are encoded in $F_A$, and the expansions of $(\beta_1^*)^2, 2 \beta_1^* \beta_5^*, (\beta_5^*)^2$ will be given by the terms of $F_A$ which are quadratic or cubic in $y_1, y_5$. 
\end{proof}

\begin{lemma} \label{matrix vanish}
Suppose $A \in \mathbb{C}^4 \otimes \wedge^2 \mathbb{C}^5$ vanishes on a 2-dimensional subspace $V \subseteq \mathbb{C}^5$.  Then $F_A(y) = 0$ for all $y \in V$.
\end{lemma}
\begin{proof}
Fix non-zero $y \in V$.  Consider the following $5 \times 5$ matrix:
\begin{equation}
\begin{pmatrix}
| & | & | & | & | \\
A_1 \, y & A_2 \, y & A_3 \, y & A_4 \, y & \bar{y}  \\
| & | & | & | & |
\end{pmatrix}^t
\end{equation}
where $\bar{y}$ denotes complex conjugation.

Its kernel contains $V \cap \{ \bar{y} \}^\perp$, hence is non-trivial.  So the determinant $F_A(y) \langle y,\bar{y} \rangle$ is zero, so $F_A(y) = 0$.
\end{proof}

We can now say exactly which quintic ring / sextic resolvent pairs are described by binary quintic forms.

\begin{theorem} \label{main theorem}
Let $(R,S)$ be a quintic ring / sextic resolvent pair, basis-free, of non-zero discriminant, with fundamental alternating map $g : \wedge^2 \tilde{S} \to \tilde{R}$.  Then, $(R,S)$ arise from some binary quintic form if and only if the following two conditions hold:
\begin{itemize}
\item $S$ has a basis $\{1,\beta_1,\ldots,\beta_5\}$ in which it is dual-digenic
\item The associated dual basis has the property $g(\beta_1^*,\beta_5^*) = 0$
\end{itemize}
\end{theorem}

\begin{proof}

The forward implication is true by Theorem \ref{sextic structure2} and because $\Phi(f)$ has top-right entry equal to 0.

The reverse implication, however, needs proof:  Let $A \in \mathbb{Z}^4 \otimes \wedge^2 \mathbb{Z}^5$ giving rise to $(R,S)$ with $S$ in the given special basis.

The condition $g(\beta_1^*, \beta_5^*) = 0$ translates to $A(e_1,e_5) = 0$, from which we see that $A$ is of the form:
\begin{equation}
\begin{pmatrix}
0 & a & b & c & 0 \\
- a & 0 & * & * & - d \\
- b & * & 0 & * & - e \\
- c & * & * & 0 & - f \\
0 & d & e & f & 0 \\
\end{pmatrix}
\end{equation}
for some $a,b,c,d,e,f \in \mathbb{Z}[t_1,t_2,t_3,t_4]$.  The entries marked $*$ will be unimportant.

By Corollary \ref{segre basis property}, we know that the terms of $F_A$ cubic in $y_1, y_5$ all vanish and that the partial derivatives of $F_A$ evaluated at $y_1 e_1 + y_5 e_5$ take the form:
\begin{equation}
\nabla F_A (y_1 e_1 + y_5 e_5) = (0, - y_1^2, - y_1 y_5, - y_5^2, 0)
\end{equation}
These properties will impose conditions on $a,b,c,d,e,f$ which will lead us to see that we are $SL_4(\mathbb{Z})$ equivalent to some $\Phi(f)$.

First, note that by Lemma \ref{matrix vanish}, the vanishing of terms of $F_A$ which are cubic in $y_1, y_5$ is already apparent.  So, the only interesting conditions are on $\nabla F_A$.

Recall the formula $F_A(y) \langle y, z \rangle = A_1 y \wedge A_2 y \wedge A_3 y \wedge A_4 y \wedge z$.  Differentiating with respect to $y_i$, we get:
\begin{IEEEeqnarray}{rCl}
\frac{\partial F_A}{\partial y_i}(y) \langle y, z \rangle + F_A(y) z_i & = & A_1 e_i \wedge A_2 y \wedge A_3 y \wedge A_4 y \wedge z + \ldots \nonumber \\
&& + A_1 y \wedge A_2 y \wedge A_3 y \wedge A_4 e_i \wedge z
\end{IEEEeqnarray}
Taking $y = y_1 e_1 + y_5 e_5$, this simplifies to:
\begin{IEEEeqnarray}{rCl}
\frac{\partial F_A}{\partial y_i}(y) \langle y, z \rangle & = & A_1 e_i \wedge A_2 y \wedge A_3 y \wedge A_4 y \wedge z + \ldots \nonumber \\
&& + A_1 y \wedge A_2 y \wedge A_3 y \wedge A_4 e_i \wedge z
\end{IEEEeqnarray}
We can explicitly compute the determinants on the right hand side here, and thus the partial derivatives.  The partial derivatives with respect to $y_1$ and $y_5$ come out as identically 0, as we expect from Lemma \ref{matrix vanish}.  The other partial derivatives are more interesting and are listed below.  We use the notation $\Delta(pqrs)$ to mean the determinant of the $4 \times 4$ matrix whose rows are given by $p, q, r, s \in \mathbb{Z}^4 \simeq \mathbb{Z}[t_1,t_2,t_3,t_4]$.
\begin{IEEEeqnarray}{rCl}
\frac{\partial F_A}{\partial y_2}(y) & = & - y_1^2 \, \Delta(abcd) + y_1 y_5 \,  [ \Delta(abdf) - \Delta(acde) ] - y_5^2 \, \Delta(adef) \\
\frac{\partial F_A}{\partial y_3}(y) & = & - y_1^2 \, \Delta(abce) + y_1 y_5 \,  [ \Delta(abef) - \Delta(bcde) ] - y_5^2 \, \Delta(bdef) \\
\frac{\partial F_A}{\partial y_4}(y) & = & - y_1^2 \, \Delta(abcf) + y_1 y_5 \,  [ \Delta(acef) - \Delta(bcdf) ] - y_5^2 \, \Delta(cdef) 
\end{IEEEeqnarray}
Hence, we need the following equations to be satisfied by $a,b,c,d$:
\begin{IEEEeqnarray}{rCl}
\Delta(abcd) & = & 1 \\
\Delta(abdf) - \Delta(acde) & = & 0 \\
\Delta(adef) & = & 0 \\
\Delta(abce) & = & 0 \\
\Delta(abef) - \Delta(bcde) & = & -1 \\
\Delta(bdef) & = & 0 \\
\Delta(abcf) & = & 0 \\
\Delta(acef) - \Delta(bcdf) & = & 0 \\
\Delta(cdef) & = & 1
\end{IEEEeqnarray}
It is not too hard to solve these by hand, but running them through any computer algebra package will effortlessly inform you that they amount to $e = - a, f = - b, \Delta(abcd) = 1$.  This is true over any field, not just over the ring $\mathbb{Z}$.

Thus, the matrix with rows $a,b,c,d$ lies in $SL_4(\mathbb{Z})$, and furthermore the transformation $\tau$ taking $(a,b,c,d)$ to $(t_3, - t_2, t_1, t_4)$ also lies in $SL_4(\mathbb{Z})$.  Thus, we can write:
\begin{equation} A' = \tau \cdot A = 
\begin{pmatrix}
0 & t_3 & - t_2 & t_1 & 0 \\
- t_3 & 0 & * & * & - t_4 \\
t_2 & * & 0 & * & t_3 \\
- t_1 & * & * & 0 & - t_2 \\
0 & t_4 & - t_3 & t_2 & 0 \\
\end{pmatrix}
\end{equation}

This matrix is almost of the form $\Phi(f)$.  We will be able to show that we can transform it into the required form by applying an element $h \in H \leq SL_5(\mathbb{Z})$, where $H$ is the subgroup of matrices of the form:

\begin{equation}
\begin{pmatrix}
1 & 0 & 0 & 0 & 0 \\
* & 1 & 0 & 0 & * \\
* & 0 & 1 & 0 & * \\
* & 0 & 0 & 1 & * \\
0 & 0 & 0 & 0 & 1
\end{pmatrix}
\end{equation}

Note: Any $h \in H$ gives rise to a change of basis of $S$; this change of basis will fix $\beta_1^*, \beta_5^*$ and $\beta_2, \beta_3, \beta_4$ mod $\mathbb{Z}$, so it will preserve the dual-digenization of $S$.

So, we are looking for $h \in H$ which will transform $A'$ into a matrix of the form $\Phi(f)$.  We just have to find $h$ which kills certain $t_i$ coefficients in the central $3 \times 3$ block of $A'$.  For example, in order to kill the $t_4$ term in $A'(e_2, e_3)$, we translate $e_3$ by a choice multiple $n$ of $e_5$, as $A'(e_2, e_3 + n e_5) = A'(e_2, e_3) - n t_4$.  These conditions amount to six linear equations in the entries of $h$, easily seen to have a unique solution; if we didn't have $A'(e_1,e_5) = 0$, these equations would be quadratic and their solubility would be unclear.
\end{proof}

\section{The action of $GL_2(\mathbb{Z})$ on the sextic resolvent ring}

Recall Theorem \ref{GL2}:

\begin{theorem} \nonumber
Let $f \in (Sym^5 \mathbb{Z}^2)^*$ and $\gamma \in GL_2(\mathbb{Z})$.  Then there exists $h \in H$ such that $\Phi( \gamma \cdot f) = (1,h) \cdot \sigma ( \gamma ) \cdot \Phi(f)$
\end{theorem}

This results in the following action on the basis of the sextic resolvent ring:
\begin{corollary} \label{GL2 ring}
Let $f \in (Sym^5 \mathbb{Z}^2)^*$, $\gamma \in GL_2(\mathbb{Z})$ and $g = \gamma \cdot f$.

Denote the basis of $S_f$ by $\{ 1, \beta_{1,f},\ldots,\beta_{5,f} \}$, with the usual notation $\beta_{i,f}^*$ for the dual basis of $\tilde{S_f}$. Denote the basis elements of $S_g$ and $\tilde{S_g}$ analogously.

Then:
\begin{IEEEeqnarray}{rCl}
\begin{pmatrix}
\beta_{1,g}^* \\ \beta_{5,g}^*
\end{pmatrix} & = &
\begin{pmatrix} d & - c \\ - b & a \end{pmatrix}
\begin{pmatrix}
\beta_{1,f}^* \\ \beta_{5,f}^*
\end{pmatrix} \\
\begin{pmatrix}
\beta_{2,g} \\ \beta_{3,g} \\ \beta_{4,g}
\end{pmatrix} & \equiv &
\begin{pmatrix} d^2 & - c d & c^2  \\ - 2 b d & b c + a d & - 2 a c \\ b^2 &- a b & a^2 \end{pmatrix}
\begin{pmatrix}
\beta_{2,f} \\ \beta_{3,f} \\ \beta_{4,f}
\end{pmatrix} \quad \text{mod} \quad \mathbb{Z}
\end{IEEEeqnarray}
\end{corollary}

\begin{proof}
The $SL_5$ component of $(1,h) \cdot \sigma ( \gamma )$ is of the form:
\begin{equation}
\begin{pmatrix}
d & 0 & 0 & 0 & - c \\
* & a^2 & 2ab & b^2 & * \\
* & ac & bc+ad & bd & * \\
* & c^2 & 2cd & d^2 & * \\
- b & 0 & 0 & 0 & a
\end{pmatrix}
\end{equation}

An element $A \in \mathbb{Z}^4 \otimes \wedge^2 \mathbb{Z}^5$ represents a map $\wedge^2 \tilde{S} \to \tilde{R}$, and so the action of $SL_5$ on this space corresponds to a change of basis of $\tilde{S}$, instead of $S$.  The reason we write almost is because an orientation condition on the pair $(R(A),S(A))$ means that $\tau \in GL_4(\mathbb{Z})$ will further change the basis of $S(A)$ by a factor of $\det \tau$.

Since the $GL_4$ component of $(1,h) \cdot \sigma ( \gamma )$ lies in $SL_4(\mathbb{Z})$, the $SL_5$ component represents the change of basis of $\tilde{S}$ on the nose.  The matrix above tells us the relation between the bases of $\tilde{S_f}$ and $\tilde{S_g}$, while the transpose inverse tells us how the bases of $S_f / \mathbb{Z}$ and $S_g / \mathbb{Z}$ are related.  The result follows.
\end{proof}

Note that this respects the dual-digenicity relation between $\{ \beta_1^*, \beta_5^* \}$ and $\{ \beta_2, \beta_3, \beta_4 \}$.

There is also a subtle point to be made here: We can view $S_f^* \otimes \mathbb{Q}$ as a 6-dimensional representation of $GL_2(\mathbb{Z})$, and likewise for $S_f \otimes \mathbb{Q}$.  From the block matrix form of $\rho (\gamma) \in GL_5(\mathbb{Z})$, we see that both these representations break into irreducible components of dimensions 1, 2 and 3.  We have an equality $S_f^* \otimes \mathbb{Q} = S_f \otimes \mathbb{Q}$ given by the trace pairing, and we might then think that the subrepresentations occuring in each representation are equal.  However, this is not the case; the key point to be noted is that, on this shared space, the two $GL_2(\mathbb{Z})$-actions are in general different, as explained by the following lemma:

\begin{lemma}
Let $K$ be a field.  Suppose that $V$ is a $K$-vector space with basis $\{v_0,v_1,\ldots,v_{n-1}\}$, and with a non-degenerate pairing $\langle -,- \rangle : V \otimes_K V \to K$.  Using $\langle -,- \rangle$, we can identify $V^*$ and $V$; denote the dual basis by $\{v_0^*,v_1^*,\ldots,v_{n-1}^*\} \subset V$.  Let $\sigma \in GL(V)$ such that $\sigma(v_i) = w_i$ and $\sigma(v_i^*) = w_i^*$, where $\{w_0^*,w_1^*,\ldots,w_{n-1}^*\}$ is the dual basis of $\{w_0,w_1,\ldots,w_{n-1}\}$.  Then $\sigma$ stabilises $\langle -,- \rangle$.
\end{lemma}

\begin{proof}
The dual basis is obtained as follows:
\begin{equation}
\begin{pmatrix}
v_0^* \\ v_1^* \\ \ldots \\ v_{n-1}^*
\end{pmatrix}
=
\begin{pmatrix}
\langle v_i,v_j \rangle
\end{pmatrix}^{-1}_{i,j}
\begin{pmatrix}
v_0 \\ v_1 \\ \ldots \\ v_{n-1}
\end{pmatrix}
\end{equation}

Since $\sigma$ is $K$-linear, we have:
\begin{equation}
\begin{pmatrix}
\sigma(v_0^*) \\ \sigma(v_1^*) \\ \ldots \\ \sigma(v_{n-1}^*)
\end{pmatrix}
=
\begin{pmatrix}
\langle v_i,v_j \rangle
\end{pmatrix}^{-1}_{i,j}
\begin{pmatrix}
\sigma(v_0) \\ \sigma(v_1) \\ \ldots \\ \sigma(v_{n-1})
\end{pmatrix}
\end{equation}

But if $\sigma(v_i) = w_i$ and $\sigma(v_i^*) = w_i^*$ then this simplifies:
\begin{equation}
\begin{pmatrix}
w_0^* \\ w_1^* \\ \ldots \\ w_{n-1}^*
\end{pmatrix}
=
\begin{pmatrix}
\langle v_i,v_j \rangle
\end{pmatrix}^{-1}_{i,j}
\begin{pmatrix}
w_0 \\ w_1 \\ \ldots \\ w_{n-1}
\end{pmatrix}
\end{equation}

However, we also know to obtain $w_i^*$ in the following way:
\begin{equation}
\begin{pmatrix}
w_0^* \\ w_1^* \\ \ldots \\ w_{n-1}^*
\end{pmatrix}
=
\begin{pmatrix}
\langle w_i,w_j \rangle
\end{pmatrix}^{-1}_{i,j}
\begin{pmatrix}
w_0 \\ w_1 \\ \ldots \\ w_{n-1}
\end{pmatrix}
\end{equation}

Hence, $\langle w_i,w_j \rangle = \langle v_i,v_j \rangle$ for all $i,j$, and so $\sigma$ stabilises the pairing.
\end{proof}

This means that we should be careful when using this representation-theoretic perspective to draw links between $S_f$ and $S_f^*$.

\section{Classes of binary quintic forms and sextic resolvent rings}

Now that we understand how $GL_2(\mathbb{Z})$ acts on the bases of the quintic ring and sextic resolvent, we will be able to fully understand classes of binary quintics through the lens of these rings.

First, we introduce two definitions which recognise the isomorphisms between rings coming from the $GL_2(\mathbb{Z})$ action:
\begin{definition}
Let $(R,S)$ be a quintic ring / sextic resolvent pair, with both rings of non-zero discriminant.

Consider the map which sends a dual-digenic basis $\mathfrak{B}$ of $S$ to the submodule $\mathbb{Z} \{ 1, \beta_2, \beta_3, \beta_4 \} \subseteq S$.  A \emph{dual-digenization} of $S$ is a non-empty preimage under this map, i.e. a class of dual-digenic bases of $S$, all producing the same sub-module via the construction $\mathbb{Z} \{ 1, \beta_2, \beta_3, \beta_4 \}$.

Equivalently, each class is determined by $\mathbb{Z} \{ \beta_1^*, \beta_5^* \}$, as this is the submodule of $S^*$ which is orthogonal to $\mathbb{Z} \{ 1, \beta_2, \beta_3, \beta_4 \}$.
\end{definition}

The key structure of a dual-digenic based ring comes from the relation between $\{ \beta_1^*, \beta_5^* \}$ and $\{ \beta_2, \beta_3, \beta_4 \}$.  However, applying an element of $GL_2(\mathbb{Z})$ to $\{ \beta_1^*, \beta_5^* \}$ gives rise to a new dual-digenic basis, so the key object is really $\mathbb{Z} \{ \beta_1^*, \beta_5^* \}$.  The definition of dual-digenization is cooked up to recognise this.

\begin{definition}
Let $S$ and $S'$ be based sextic rings of non-zero discriminant, with dual-digenic bases.  A \emph{dual-digenic isomorphism} between $S$ and $S'$ is an isomorphism which respects the dual-digenizations of the two rings.
\end{definition}

\begin{corollary} \label{GL2 dual digenic}
Let $f$ be a binary quintic form of non-zero discriminant, let $\gamma \in GL_2(\mathbb{Z})$ and $g = \gamma \cdot f$.  Then 
there is a dual-digenic isomorphism from $S_f$ to $S_g$.
\end{corollary}
\begin{proof}
Theorem \ref{GL2} implies that $S_f$ and $S_g$ are isomorphic.  Corollary \ref{GL2 ring} details this isomorphism and makes clear that it preserves the dual-digenizations of $S_f$ and $S_g$.
\end{proof}

We can now state our main theorem relating classes of binary quintic forms to quintic rings and sextic resolvent rings:
\begin{theorem} \label{main class theorem}
There is a bijection between the following two sets:
\begin{equation}
GL_2(\mathbb{Z}) \backslash (Sym^5 \mathbb{Z}^2)^* \leftrightarrow
\left \{
\begin{array}{l}
(R,S), \text{R basis-free} \\
\text{S dual-digenic with a} \\
\text{fixed dual-digenization} \\
\text{and } g(\beta_1^*,\beta_5^*) = 0 
\end{array}
\right \}
 / \sim
\end{equation}
given by $f \mapsto$ the class of $(R_f, S_f)$, where the $\sim$ denotes isomorphism of $R$ and dual-digenic isomorphism of $S$, and where $g$ is the fundamental alternating map $\wedge^2 \tilde{S} \to \tilde{R}$.
\end{theorem}

\begin{proof}
From Theorem \ref{main theorem}, we know that every binary quintic form gives rise to a sextic resolvent ring $S_f$ with a dual-digenic basis, with $g(\beta_1^*,\beta_5^*) = 0$, and furthermore by Corollary \ref{GL2 dual digenic} the map above descends to classes of forms.

Conversely, let $(R,S)$ be a quintic ring / sextic resolvent pair, with a fixed dual-digenization of $S$ and $g(\beta_1^*, \beta_5^*) = 0$.  Theorem \ref{main theorem} also tells us that there are bases of $R$ and $S$ such that $(R,S)$ is given by $\Phi(f)$ for some $f$.  The change of basis of $S$ comes from $h \in H$, so it preserves the choice of dual-digenization.  Hence, the map in the statement of this theorem is surjective.

For injectivity, suppose $(R_f,S_f)$ and $(R_g,S_g)$ are isomorphic, with a dual-digenic isomorphism of $S_f$ and $S_g$.  Because $\mathbb{Z} \{ \beta_1^*, \beta_5^* \}$ is fixed, one component of the change of basis from $\tilde{S_f}$ to $\tilde{S_g}$ looks like:
\begin{IEEEeqnarray}{rCl}
\begin{pmatrix}
\beta_{1,g}^* \\ \beta_{5,g}^*
\end{pmatrix} & = &
\begin{pmatrix} d & - c \\ - b & a \end{pmatrix}
\begin{pmatrix}
\beta_{1,f}^* \\ \beta_{5,f}^*
\end{pmatrix}
\end{IEEEeqnarray}
for some matrix $\begin{pmatrix} d & - c \\ - b & a \end{pmatrix} \in GL_2(\mathbb{Z})$.

Recall from Lemma \ref{recovery} that $Tr ( (u \beta_{5,f}^* - v \beta_{1,f}^*)^5) = - 10 f(u,v)$, and similarly for $g$.  It follows that $f$ and $g$ are $GL_2(\mathbb{Z})$-equivalent, as desired.
\end{proof}

\section{The Cayley-Klein resolvent map}

Recall from Higher Composition Laws IV the Cayley-Klein resolvent map $\psi: R \to \tilde{S}$, defined as
\begin{IEEEeqnarray}{rCll}
\alpha & \mapsto & \frac{1}{\sqrt{Disc(R)}} & (\alpha^{(1)} \alpha^{(2)} + \alpha^{(2)} \alpha^{(3)} + \alpha^{(3)} \alpha^{(4)} + \alpha^{(4)} \alpha^{(5)} + \alpha^{(5)} \alpha^{(1)} \nonumber \\
&&& - \alpha^{(1)} \alpha^{(3)} - \alpha^{(3)} \alpha^{(5)} - \alpha^{(5)} \alpha^{(2)} - \alpha^{(2)} \alpha^{(4)} - \alpha^{(4)} \alpha^{(1)}) \nonumber \\
\end{IEEEeqnarray}

\begin{lemma}
The Cayley-Klein map $\psi : R_f \to \tilde{S_f}$ has the following property:
\begin{equation}
\psi(x^3 \alpha_1 + x^2 y \alpha_2 + x y^2 \alpha_3 + y^3 \alpha_4) = 4 f(x,y) ( y \beta_1^* - x \beta_5^*)
\end{equation}
\end{lemma}

\begin{proof}
Bhargava shows that, for $A \in \mathbb{Z}^4 \otimes \wedge^2 \mathbb{Z}^5$ and the associated based quintic ring $R = R(A)$, this map is given by the $4 \times 4$ signed sub-Pfaffians $(P_1,\ldots,P_5)$ of $A$ as follows:
\begin{equation}
\psi(t_1 \alpha_1 + \ldots + t_4 \alpha_4) = 4 P_1(t) \beta_1^* + \ldots + 4 P_5(t) \beta_5^*
\end{equation}

Recall that when $A = \Phi(f)$, the sub-pfaffians are in the span of the following five quadrics:
\begin{IEEEeqnarray}{rCl}
Q_1 & = & t_1 t_3 - t_2^2\\
Q_2 & = & t_1 t_4 - t_2 t_3\\
Q_3 & = & t_2 t_4 - t_3^2\\
Q_4 & = &  f_0 t_1 t_2 +  f_1 t_2^2 +  f_2 t_2 t_3 +  f_3 t_3^2 +  f_4 t_3 t_4 +  f_5 t_4^2\\
Q_5 & = &  f_0 t_1^2 +  f_1 t_1 t_2 +  f_2 t_2^2 +  f_3 t_2 t_3 +  f_4 t_3^2 +  f_5 t_3 t_4
\end{IEEEeqnarray}

In fact, the five $4 \times 4$ signed sub-pfaffians are easily seen to be:
\begin{equation}
Q(A) = [Q_4, Q_1, Q_2, Q_3, - Q_5]
\end{equation}

When $A = \Phi(f)$ and $(t_1,\ldots,t_4) = (x^3, x^2 y, x y^2, y^3)$, because of the special form of the sub-Pfaffians, they simplify to give the stated result.
\end{proof}

This mimics the behaviour of the resolvent maps for binary cubics and quartics:
\begin{itemize}
\item For a binary cubic, the resolvent map has the property:
\begin{equation}
x \alpha_1 + y \alpha_2 \mapsto f(x,y) \omega
\end{equation}
where $\omega$ generates the quadratic resolvent ring.

\item For a binary quartic, the resolvent map has the property:
\begin{equation}
x^2 \alpha_1 + x y \alpha_2 + y^2 \alpha_3 \mapsto f(x,y) \omega
\end{equation}
where $\omega$ generates the cubic resolvent ring.
\end{itemize}

So, it seems that the resolvent map picks out some of the key structure of the resolvent ring, when evaluated on a certain family of elements corresponding to the relevant rational normal curve.

\chapter{Counting binary quintics of bounded discriminant}

From Theorem \ref{main class theorem}, we understand how classes of binary quintic forms map into the space $\mathbb{Z}^4 \otimes \wedge^2 \mathbb{Z}^5$.  Results counting $\Gamma$-classes of bounded discriminant in $\mathbb{Z}^4 \otimes \wedge^2 \mathbb{Z}^5$ exist [source?], and we would like to port these over to give counting results on binary quintics.  The one missing ingredient is our understanding of how many $GL_2(\mathbb{Z})$-classes of binary quintics can map into one $\Gamma$-orbit, which is the principal problem we aim to tackle here.  We will explain geometric and algebraic approaches to this problem.

\section{A geometrical viewpoint}

\subsection{Lines on the Segre cubic}

Let $A \in \mathbb{Z}^4 \otimes \wedge^2 \mathbb{Z}^5$ and $S = S(A)$.  Recall from Corollary \ref{segre basis property} that dual-digenicity of $S$ is reflected in the equation $F_A$ of the associated Segre cubic.  Specifically, the based ring $S$ is dual-digenic if and only if $F_A$ takes the form $F_A(y) = - y_1^2 y_2 - y_1 y_3 y_5 - y_4 y_5^2 + \text{(lower order terms in $y_1, y_5$)}$.  In particular there is a distinguished line $L = sp\{e_1,e_5\}$ on the Segre cubic.

We will see that all dual-digenizations of the ring $S$ have associated lines on the Segre cubic and that the ones arising from binary quintics, meaning that also $g(\beta_1^*, \beta_5^*) = 0$, actually come from a special component of the Fano variety of lines on the Segre cubic.

\begin{lemma}
Let $A \in \mathbb{Z}^4 \otimes \wedge^2 \mathbb{Z}^5$ of non-zero discriminant, and let $S = S(A)$.  Consider $S$ as a basis-free ring.  Then, each dual-digenization of $S$ (if any exist) gives rise to a distinct line on the Segre cubic associated to $A$, which we denote by $V_A$.
\end{lemma}

\begin{proof}
We know that $A$ gives rise to a dual-digenic basis of $S$ if and only if $F_A(y) = - y_1^2 y_2 - y_1 y_3 y_5 - y_4 y_5^2 + \text{(lower order terms in $y_1, y_5$)}$.  In this situation, we also then have the weaker statement that $F_A$ vanishes on $sp\{e_1,e_5\}$.

Now, let's consider different bases of $S$.  Lemma \ref{segre action} tells us that if $\gamma \in GL_5(\mathbb{C})$ and $A' = \gamma \cdot A$ then:
\begin{equation}
F_{A'}(y) = \det(\gamma) \, F_A(\gamma^t y)
\end{equation}

Denote the $i$-th row of $\gamma$ by $\gamma_i$.  If $A'$ gives rise to a dual-digenic basis of $S$, $F_{A'}$ vanishes on $sp\{e_1,e_5\}$, and $F_A$ vanishes on $sp\{\gamma_1,\gamma_5\}$.

We need to see that this construction gives the same line for two bases precisely when they are in the same dual-digenization.  Recall that two dual-digenic bases are in the same dual-digenization if $\mathbb{Z}\{\beta_1^*,\beta_5^*\}$ is the same for both of them.  The rows of $\gamma$ give the change of dual basis, so taking a different basis in the same dual-digenization means acting on $( \gamma_1, \gamma_5 )$ by some element of $GL_2(\mathbb{Z})$.  Such a transformations keeps the line constructed above fixed.  Conversely, if $\gamma, \gamma'$ give rise to the same line, then $(\gamma_1,\gamma_5)$ and $(\gamma'_1, \gamma'_5)$ are related by an element of $GL_2(\mathbb{Z})$, as they are $\mathbb{Z}$-bases of the maximal lattice contained in the line.  If they both give rise to dual-digenic bases, they are therefore in the same dual-digenization.
\end{proof}

This means that each dual-digenization of $S$ gives rise to a unique point in the Fano variety of lines on $V_A$.  Dolgachev proves that this variety is a 2-dimensional variety given by 15 planes and 6 degree five del Pezzo surfaces, and gives a description of the lines represented by each component.  In particular, each of its irreducible components has infinitely many rational points.  We know there are finitely many classes of forms of the same discriminant, so only finitely many of these lines can be of note to us.  There must be other equations to cut down the search, and at some point we need to work over $\mathbb{Z}$.

The next step is to show that the lines arising from binary quintics all come from a distinguished one of these del Pezzo surfaces, which Dolgachev calls $D_6$.

\subsection{The del Pezzo surface $D_6$}

There is a quadratic map $\phi_f : \mathbb{P}^3 \to V(\Phi(f))$ given by the sub-pfaffians of $\Phi(f)$.  Its indeterminacy locus is 5 points $\{p_1,\ldots,p_5\}$, which is actually the set $X_{R_f}$ coming from the quintic ring $R_f$.  Given a twisted cubic $C$ in $\mathbb{P}^3$ passing through these 5 points, the image of $C$ under this map will be a line on the Segre cubic.  The component $D_6$ of the Fano variety is precisely all such lines.

For $\Phi(f)$, this quadratic map $\phi_f$ is given by the following forms:
\begin{IEEEeqnarray}{rCl}
Q_1 & = &  f_0 t_1 t_2 +  f_1 t_2^2 +  f_2 t_2 t_3 +  f_3 t_3^2 +  f_4 t_3 t_4 +  f_5 t_4^2\\
Q_2 & = & t_1 t_3 - t_2^2\\
Q_3 & = & t_1 t_4 - t_2 t_3\\
Q_4 & = & t_2 t_4 - t_3^2\\
Q_5 & = &  - (f_0 t_1^2 +  f_1 t_1 t_2 +  f_2 t_2^2 +  f_3 t_2 t_3 +  f_4 t_3^2 +  f_5 t_3 t_4)
\end{IEEEeqnarray}

Note that the rational normal curve in $\mathbb{P}^3$, cut out by the vanishing of $\{Q_2,Q_3,Q_4\}$, has image $\mathbb{C} \{e_1,e_5\}$, and that the indeterminacy locus of this map lies on the rational normal curve.  So this line is represented by a point of $D_6$.  In fact, all the dual-digenizations coming from binary quintic forms will give rise to points of $D_6$; however, this is not the case for any dual-digenization, as we will see later.

\begin{lemma}
Let $f,g$ be binary quintic forms of non-zero discriminant, with $(\tau,\gamma) \in GL_4(\mathbb{Z}) \times GL_5(\mathbb{Z})$ such that $\Phi(g) = (\tau,\gamma) \cdot \Phi(f)$.  Then, the line on $V_{\Phi(f)}$ coming from the dual-digenic basis given by $\gamma$ is represented by a point of $D_6$.
\end{lemma}

\begin{proof}
We have seen that we have a line on the Segre cubic $V_{\Phi(g)}$ which is the image of a twisted cubic; namely, the line $\mathbb{C}\{e_1, e_5\}$ which is the image of the rational normal curve.

The maps $\mathbb{P}^3 \to V_{\Phi(f)}$ and $\mathbb{P}^3 \to V_{\Phi(g)}$ are related by $(\tau,\gamma)$ via an action on the sub-pfaffians, described in Lemma \ref{sub-pfaffian action} (action on sub-pfaffians) which was proved in Higher Composition Laws IV.  The following commutative diagram summarises the relation:
\begin{equation}
\begin{tikzcd}
\mathbb{P}^3 \arrow[r, "\phi_f"] & V_{\Phi(f)} \\
\mathbb{P}^3 \arrow[r, "\phi_g"] \arrow[u, "\tau^t"] & V_{\Phi(g)} \arrow[u, "\gamma^t"]
\end{tikzcd}
\end{equation}

To use this diagram to our benefit, we trace the image of the rational normal curve in the bottom $\mathbb{P}^3$ through the various maps.  Its image in the top $\mathbb{P}^3$ is a twisted cubic.  Furthermore, this curve goes through the five points of $X_{R_f}$; this is because the rational normal curve goes through $X_{R_g}$ and $X_{R_f} = \tau^t X_{R_g}$.  Meanwhile, the image of the rational normal curve under $\phi_g$ is the line $\mathbb{C}\{e_1, e_5\} \subseteq V_{\Phi(g)}$, whose image in $V_{\Phi(f)}$ under $\gamma^t$ is the line $\mathbb{C}\{\gamma_1,\gamma_5\}$.  So, the line on $V_{\Phi(f)}$ coming from $\gamma$ is represented by a point of $D_6$, according to Dolgachev's description.
\end{proof}

We have narrowed the problem down to searching for rational points on this component $D_6$.  However, a degree five del Pezzo surface is birational to $\mathbb{P}^2$ over $\mathbb{Q}$, so this component still has infinitely many rational points!

Furthermore, our problem concerns integral binary quintics, not rational binary quintics, so we need to move our problem over to $\mathbb{Z}$ by dehomogenising $D_6$, and then find a subvariety of this that captures only the finitely many points represented by classes of integral binary quintic forms.  As we move to $\mathbb{Z}$, the problem will take on a more algebraic flavour.

But, before we move on to these problems, let us understand the other components of the Fano variety a little better.

\subsection{The other components of the Fano variety}

First of all, the 15 planes of the Fano variety are the lines contained in the 15 planes of the Segre cubic.

\begin{lemma} \label{not plane}
Let $A \in \mathbb{Z}^4 \otimes \wedge^2 \mathbb{Z}^5$ of non-zero discriminant, and let $S = S(A)$.  A line arising from a dual-digenization of $S$ does not lie on a plane of the Segre cubic.
\end{lemma}
\begin{proof}
Let $L$ be a line on the Segre cubic $V_A$ arising from a dual-digenization of $S$.  We can find $(\tau, \gamma) \in GL_4(\mathbb{Z}) \times GL_5(\mathbb{Z})$ such that the basis of $S$ arising from $A' = (\tau, \gamma) \cdot A$ is in this dual-digenization.  Thus, $L = \mathbb{C}\{\gamma_1,\gamma_5\}$.  Recall from Lemma \ref{segre basis property} that:
\begin{equation}
F_{A'}(y) = - y_1^2 y_2 - y_1 y_3 y_5 - y_4 y_5^2 + \text{(lower order terms in $y_1, y_5$)}
\end{equation}
This can be rephrased as:
\begin{equation}
\nabla F_{A'}(y_1 e_1 + y_5 e_5) = (0, - y_1^2, - y_1 y_5, - y_5^2, 0)
\end{equation}
Also, recall from Lemma \ref{segre action} that:
\begin{equation}
F_{A'}(y) = \det(\tau) \det(\gamma) F_A(\gamma^t y)
\end{equation}
Differentiating this equation gives:
\begin{equation}
\nabla F_{A'} (y) = \det(\tau) \det(\gamma) \nabla F_A (\gamma^t y) \cdot \gamma^t
\end{equation}
When we evaluate this last equation at $y = y_1 e_1 + y_5 e_5$ and apply $(\gamma^t)^{-1}$, we see that:
\begin{equation}
\text{dim(sp} \{ \nabla F_{A} ( l ) : l \in L \}) = 3
\end{equation}
However, if $L$ were contained in some plane $P$ then $\{ \nabla F_A ( l ) : l \in L \}) \subseteq P^\perp$, hence its dimension would be bounded by 2.
\end{proof}

The other five del Pezzos, $D_1,\ldots,D_5$, are given as follows:  take $p_i$ in the indeterminacy locus of the map $\mathbb{P}^3 \to V(\Phi(f))$.  The image under this map of a line passing through $p_i$ is a line on the Segre cubic.  The component $D_i$ is the union of such lines.

\begin{lemma} \label{not D1,...,D5}
Let $f,g$ be binary quintic forms of non-zero discriminant, with $(\tau,\gamma) \in GL_4(\mathbb{Z}) \times GL_5(\mathbb{Z})$ such that $\Phi(g) = (\tau,\gamma) \cdot \Phi(f)$.  Then, the line on $V_{\Phi(f)}$ coming from the dual-digenic basis given by $\gamma$ is not represented by a point of $D_1,\dots,D_5$.
\end{lemma}
\begin{proof}
Start with the simplest case: the line $L = \mathbb{C}\{e_1,e_5\}$ on $V_{\Phi(f)}$.  By looking at the five sub-pfaffians giving $\phi_f$, it is easy to see that the preimage of $L$ under this map is the rational normal curve.  This curve contains no lines and so $L$ does not lie in any of the del Pezzos $D_1, \ldots, D_5$.

Consider the following diagram again:
\begin{equation}
\begin{tikzcd}
\mathbb{P}^3 \arrow[r, "\phi_f"] & V_{\Phi(f)} \\
\mathbb{P}^3 \arrow[r, "\phi_g"] \arrow[u, "\tau^t"] & V_{\Phi(g)} \arrow[u, "\gamma^t"]
\end{tikzcd}
\end{equation}

Now let $L$ be the line on $V_{\Phi(f)}$ coming from $\gamma$, i.e. $L = \mathbb{C}\{\gamma_1, \gamma_5\}$.  This diagram shows us that:
\begin{IEEEeqnarray}{rCl}
\phi_f^{-1}(L) & = & \tau^t \phi_g^{-1} (\mathbb{C}\{e_1, e_5\}) \\
& = &\tau^t V_3
\end{IEEEeqnarray}
where $V_3$ is the rational normal curve in $\mathbb{P}^3$.  Hence, $L$ is not the image under $\phi_f$ of any line in $\mathbb{P}^3$, and so it is not represented by a point of $D_1, \ldots, D_5$.
\end{proof}

This means that all the lines in $V_{\Phi(f)}$ of interest to us (i.e. those coming from binary quintic forms) lie in only the del Pezzo component $D_6$ of the Fano variety.

\section{An algebraic viewpoint}

Now that we understand the geometry of the situation, we would like to find equations pinning down precisely those points of $D_6$ corresponding to classes of binary quintic forms.  First of all, we'll find equations cutting out $D_6$, and then we'll dehomogenise our equations so we can truly work over $\mathbb{Z}$ and introduce further equations satisfied by dual-digenic bases.

Before we prove these results, we note that the Fano variety lives inside $Gr(2,5)$, which is the space of lines in $\mathbb{P}^4$.  Let $L = \mathbb{C}\{x,y\}$ be a line in $\mathbb{P}^4$.  We attach to $L$ its Pluecker coordinates $p_{ij} = x_i y_j - x_j y_i$ for $i < j$.  This gives a well-defined map $Gr(2,5) \to \mathbb{P}^9$, which is known as the Pluecker embedding.  The image is a dimension 6 subvariety of $\mathbb{P}^9$ cut out by the 5 equations of the form $p_{ij} p_{kl} + p_{jk} p_{il} + p_{ki} p_{jl} = 0$, called the Pluecker relations.  Using algebraic geometry, it is possible to calculate that this is a dimension 6, degree 5 variety in $\mathbb{P}^9$.

\begin{prop}
Let $f$ be a binary quintic form of non-zero discriminant. Let $L$ be a line in $\mathbb{P}^3$ on which $\Phi(f)$ vanishes identically.  Then $L$ is a line on the Segre cubic $V_{\Phi(f)}$, and the points of $D_6$ correspond precisely to all such lines.

Hence, $D_6$ is cut out by the following equations inside $Gr(2,5)$, which come from the equation $\Phi(f)(x,y) = 0$:
\begin{IEEEeqnarray}{rCl}
p_{14} - f_0 p_{23} & = & 0 \\
- p_{45} - p_{13} - f_1 p_{23} - f_2 p_{24} & = & 0 \\
p_{12} + p_{35} - f_3 p_{24} - f_4 p_{34} & = & 0 \\
- p_{25} - f_5 p_{34} & = & 0
\end{IEEEeqnarray}
\end{prop}

\begin{proof}
Let $L = \mathbb{C}\{x,y\}$, with $\Phi(f)(x,y) = 0$.  The line $L$ lies on the Segre cubic by Lemma \ref{matrix vanish}.  Furthermore, the property of $\Phi(f)$ vanishing on $L$ is independent of the basis of $L$, so it should be given by equations in the Pluecker coordinates $p_{ij}$.  This is seen to be the case by looking at each $t_i$ component of $\Phi(f)(x,y) = 0$, and this is how the four equations above enter the picture.

The Pluecker relations and these linear equations combined cut out a subvariety of $\mathbb{P}^9$, which we denote by $W$.  Since the image of the Pluecker embedding has dimension 6, intersecting it with four hyperplanes results in a (possibly reducible) variety with irreducible components of dimension at least 2.  However, we know that $W$ is a subvariety of the 2-dimensional Fano variety of the Segre cubic, so the dimension of each irreducible component is precisely 2.  Intersecting an irreducible variety with a hyperplane preserves degree, so every irreducible component of $W$ has dimension 2 and $W$ has degree 5 (with multiplicity).  Since its degree is 5, it is either one of the del Pezzo surfaces, or a union of at most five planes.

Up until now, everything was true for a generic $A \in \mathbb{Z}^4 \otimes \wedge^2 \mathbb{Z}^5$.  At this point, we use something about the shape of $\Phi(f)$.  Namely, $\Phi(f)$ always has top-right entry 0, so the line $L = sp \{e_1, e_5 \}$ - given by $p_{15} = 1, p_{ij} = 0$ for all other $(i,j)$ - is in $W$.  However, we saw that it is not contained in any of the planes or any of the del Pezzo surfaces $D_1, \ldots, D_5$.  Hence, $W = D_6$.
\end{proof}

Note: This should actually be true for any non-degenerate $A \in \mathbb{C}^4 \otimes \mathbb{C}^5$, because these $A$ form a single $GL_4 \times GL_5$ orbit, and the following sets of components of the Fano variety are invariant under the action of this group:
\begin{itemize}
\item The 15 components coming from lines through the planes of the Segre cubic
\item $D_1, \ldots, D_5$, as these come from lines in $\mathbb{P}^3$ passing through at least one point of $X_{R(A)}$
\item $D_6$, as this component comes from twisted cubics in $\mathbb{P}^3$ through all of $X_{R(A)}$
\end{itemize}

Similarly, Lemma \ref{not D1,...,D5} should generalise to the case where $A$ and $B$ are equivalent matrices having top-right entries being 0 and with $S(A),S(B)$ having dual-digenic bases.

Our next step is to introduce stricter equations for dual-digenicity of a basis.  Previously, we only used that the Segre cubic has no terms cubic in $y_1, y_5$ in this case, but we did not use the full property of Lemma \ref{segre basis property}.  This is genuinely an integral property, so it has the added bonus of finally moving our equations over $\mathbb{Z}$.

Let $A' = (\tau,\gamma) \cdot A$, with $S(A')$ having a dual-digenic basis.  We know from Lemma \ref{segre basis property} and from the proof of Lemma \ref{not plane} that:
\begin{IEEEeqnarray}{rCl}
\nabla F_{A'}(y_1 e_1 + y_5 e_5) & = & (0, - y_1^2, - y_1 y_5, - y_5^2, 0) \\
\nabla F_{A'} (y) & = & s \cdot \nabla F_A (\gamma^t y) \cdot \gamma^t
\end{IEEEeqnarray}
where $s =\det(\tau) \det(\gamma) \in \{ \pm 1 \}$ for shorthand.

Thus, the top equation, which encodes dual-digenicity, is true if and only if:
\begin{IEEEeqnarray}{rCrCl}
& & (0, - y_1^2, - y_1 y_5, - y_5^2, 0) & = & s \cdot \nabla F_A (y_1 \gamma_1 + y_5 \gamma_5) \cdot \gamma^t \\
\iff & \begin{cases} - e_2 \\ - e_2 - e_3 - e_4 \\ - e_4 \end{cases} & 
\end{IEEEeqnarray}

\begin{lemma}
Let $f$ be a binary quintic form of non-zero discriminant.  Suppose $L$ is a line on $V_{\Phi(f)}$ and is represented by a point on $D_6$. Then, $L$ arises from a dual-digenic change of basis if and only if the following equations hold:
\end{lemma}

\end{document}